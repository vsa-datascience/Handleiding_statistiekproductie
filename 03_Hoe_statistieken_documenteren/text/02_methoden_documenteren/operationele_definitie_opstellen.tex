



Een kleine tip om de operationele definitie duidelijk uit te schrijven: Maak slim gebruik van voorzetsels zoals ``voor'', ``in'' of ``per'' om de attributen en dimensies aan te duiden. Zo kan je bijvoorbeeld een cijfertabel definiëren als het ``aantal inwoners IN het jaar 2019 PER Vlaamse gemeente (volgens de NIS-code-indeling van 2019), VOOR elk geslacht (mannen, vrouwen, totaal), en PER leeftijdsgroep (0-18, 19-30, 31-65, 66+ jaar, totaal).'' Pas hierbij goed op dat je duidelijk bent bij percentages. ``Het percentage inwoners per geslacht in elke gemeente'' is niet hetzelfde als het ``het percentage inwoners per gemeente voor elk geslacht''. 