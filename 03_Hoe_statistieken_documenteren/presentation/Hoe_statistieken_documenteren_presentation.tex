% !TEX program = xelatex

\documentclass{beamer}
\usetheme[SV]{VSA}

\usepackage{tikz}
\usetikzlibrary{shadows.blur}
\tikzset{shadow/.style={blur shadow={shadow blur steps=100,shadow xshift=5pt,shadow yshift=-5pt}}}

\usepackage{setspace}


\begin{document}



\title{Hoe statistieken documenteren}
\author{Jorre Vannieuwenhuyze}
\date{}
%\date{13 oktober 2025}
%\institute{Stuurgroep transitietraject}

\titleframe





\begin{frame}
\frametitle{We maken een metadatafiche\\per tabelreeks}
De fiche bevat informatie over 
\begin{itemize}
\item gebruikersbehoefte $\rightarrow$ conceptuele definitie
\item operationele definitie
\item ontsluiting
\item kwaliteitsevaluatie
\item administratieve informatie (eigenaarschap, contact,\ldots)
\end{itemize}
\end{frame}




\section*{Gebruikersbehoeften}



\begin{frame}
\frametitle{Gebruikersbehoeften zijn de start\\van elk statistiekproductieproces}
\hfill
\begin{tikzpicture}[
	x=4cm,
	y=-2.5\baselineskip,
	arrow/.style={softsteelblue,-latex,rounded corners,line width=1pt},
	arrowdot/.style={arrow,line width=.2pt},
	]
\node at (0,0) (A) {Gebruikersbehoeften};
\node at (0,1) (B) {Conceptuele definitie};
\node at (0,2) (C) {Operationele definitie};
\node at (0,3) (D) {Ontsluiting};
\node at (1,4) (E) {Kwaliteit};
\draw[arrow] (A) -- (B);
\draw[arrow] (B) -- (C);
\draw[arrow] (C) -- (D);
\draw[arrowdot] (A) -| (E);
\draw[arrowdot] (B) -| (E);
\draw[arrowdot] (C) -| (E);
\draw[arrowdot] (D) -| (E);
\end{tikzpicture}
\end{frame}




\begin{frame}
\frametitle{SIMS verwacht documentatie\\over gebruikersbehoeften}
\hfill
\begin{tikzpicture}
\node[draw=charcoalgray,fill=white,shadow,inner sep=5mm]{\includegraphics[width=90mm]{figures/SIMS_userneeds.png}};
\end{tikzpicture}
\end{frame}





\begin{frame}
\frametitle{Gebruikersbehoeften\\komen uit verschillende hoeken}
\begin{center}
\begin{spacing}{1.5}
Internationale verplichting \\
Institutionele vraag \\
Beleidsondersteunende vraag \\
Maatschappelijke vraag | Academische vraag
\end{spacing}
\end{center}
\end{frame}





\begin{frame}
\frametitle{Gebruikersbehoeften verwijzen naar\\internationale verplichtingen}
\hfill
\begin{tikzpicture}
\node[shadow,inner sep=0pt]{\includegraphics[width=90mm]{figures/Richtlijn2008-98-EG.png}};
\end{tikzpicture}
\end{frame}

\begin{frame}
\frametitle{Gebruikersbehoeften verwijzen naar\\institutionele vragen}
\hfill
\begin{tikzpicture}
\node[shadow,inner sep=0pt]{\includegraphics[width=90mm]{figures/VCRO.png}};
\end{tikzpicture}
\end{frame}

\begin{frame}
\frametitle{Gebruikersbehoeften verwijzen naar\\beleidsondersteunende vragen}
\hfill
\begin{tikzpicture}
\node[shadow,inner sep=0pt]{\includegraphics[width=90mm]{figures/BRV.png}};
\end{tikzpicture}
\end{frame}

\begin{frame}
\frametitle{Gebruikersbehoeften verwijzen naar\\maatschappelijke vragen}
\hfill
\begin{tikzpicture}
\node[shadow,inner sep=0pt]{\includegraphics[width=90mm]{figures/krant.png}};
\end{tikzpicture}
\end{frame}

\begin{frame}
\frametitle{Gebruikersbehoeften verwijzen naar\\maatschappelijke vragen}
\hfill
\begin{tikzpicture}
\node[shadow,inner sep=0pt]{\includegraphics[width=90mm]{figures/parlementairevraag.png}};
\end{tikzpicture}
\end{frame}

\begin{frame}
\frametitle{Gebruikersbehoeften verwijzen naar\\academische vragen}
\hfill
\begin{tikzpicture}
\node[shadow,inner sep=0pt]{\includegraphics[height=65mm]{figures/sciencearticle.pdf}};
\end{tikzpicture}
\end{frame}




\begin{frame}
\frametitle{Gebruikersbehoeften leiden tot\\conceptuele definitie cijfertabellen}
VOS ``Bevolking --- omvang en groei''
\begin{itemize}
\item Aantal inwoners per gemeente per jaar
\item Bevolkingsgroei per gemeente per jaar
\end{itemize}
\end{frame}














\end{document}




