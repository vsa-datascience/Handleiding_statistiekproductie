% !TEX program = xelatex

\documentclass{beamer}
\usetheme[SV]{VSA}

\usepackage{tikz}
\usetikzlibrary{svg.path}
\usetikzlibrary{calc}
\usepackage{anyfontsize}
\usepackage{changepage}

\tikzset{
  pics/hourglass/.style={
    code={
      \pgftransformshift{\pgfpoint{-12}{-12}} % move center to (0,0)
      \path[fill=#1]
        svg "M18.25 19.25h-.66c-.38-1.36-1.55-4.74-4.4-7.25
             c2.85-2.51 4.02-5.89 4.4-7.25h.66
             c.41 0 .75-.34.75-.75s-.34-.75-.75-.75H5.75
             c-.41 0-.75.34-.75.75s.34.75.75.75h.66
             c.38 1.36 1.55 4.74 4.4 7.25
             c-2.85 2.51-4.02 5.89-4.4 7.25h-.66
             c-.41 0-.75.34-.75.75s.34.75.75.75h12.5
             c.41 0 .75-.34.75-.75s-.34-.75-.75-.75ZM7.98 4.75h8.03
             c-.45 1.42-1.6 4.25-4.02 6.28
             c-2.42-2.04-3.57-4.86-4.02-6.28ZM12 12.97
             c2.42 2.04 3.57 4.86 4.02 6.28H7.98
             c.45-1.42 1.6-4.25 4.02-6.28Z";
    }
  },
  pics/hourglass/.default=black
}




\begin{document}


\title{Basisconcepten\\openbare statistiekproductie}
\author{Jorre Vannieuwenhuyze}
\date{}
%\date{13 oktober 2025}
%\institute{Stuurgroep transitietraject}

\titleframe


\begin{frame}
\begin{quote}
\makebox[0pt][r]{``}Statistiek Vlaanderen is het netwerk van Vlaamse overheidsinstanties die \textcolor{softsteelblue}{openbare statistieken} ontwikkelen, produceren en publiceren.''
\end{quote}
\end{frame}


\begin{frame}
\frametitle{\onslide<5->{Een wetenschappelijk artikel\\volgt doorgaans een vast patroon.}}
\begin{center}
\begin{tikzpicture}[y=\baselineskip]
\onslide<5->{ \pic[scale=10,anchor=center] at (0,5.5) {hourglass=lightskyblue}; }
\onslide<1->{ \node at (0,11) {probleemstelling}; }
\onslide<1->{ \node at (0,10) {onderzoeksvraag}; }
\onslide<2->{ \node at (0,9) {$\downarrow$}; }
\onslide<2->{ \node at (0,8) {conceptuele definities}; }
\onslide<3->{ \node at (0,7) {$\downarrow$}; }
\onslide<3->{ \node at (0,6) {methode}; }
\onslide<3->{ \node at (0,5) {operationale definities}; }
\onslide<3->{ \node at (0,4) {$\downarrow$}; }
\onslide<3->{ \node at (0,3) {resultaten}; }
\onslide<4->{ \node at (0,2) {$\downarrow$}; }
\onslide<4->{ \node at (0,1) {discussie}; }
\onslide<4->{ \node at (0,0) {conclusie}; }
\end{tikzpicture}
\end{center}
\end{frame}




\begin{frame}
\frametitle{De GSBPM biedt het werkkader\\voor een statistiekproductieproces.}
%\includegraphics[width=\linewidth]{pictures/GSBPM.pdf}
\begin{adjustwidth}{-9mm}{-9mm}
\begin{tikzpicture}[
	x=.125\linewidth,
	y=-8mm,
	contentbox/.style={draw=lightskyblue,fill=lightskyblue!50,inner sep=0pt,text width=.115\linewidth,text height=7mm},
	contenttext/.style={align=center,font=\fontsize{4}{4}\selectfont,text width=.110\linewidth},
	titlebox/.style={contentbox,draw=softsteelblue,fill=softsteelblue},
	titletext/.style={contenttext,text=white,font=\fontsize{5}{5}\selectfont\bfseries},
	]
\foreach \x/\text in {
    1/{1.\\Behoeften Specificeren},
    2/{2.\\Ontwerpen},
    3/{3.\\Bouwen},
    4/{4.\\Verzamelen},
    5/{5.\\Verwerken},
    6/{6.\\Analyseren},
    7/{7.\\Verspreiden},
    8/{8.\\Evalueren}
    }{
    \node[titlebox] at (\x,0) {};
    \node[titletext] at (\x,0) {\text};
    }
\foreach \x/\y/\text in {
    1/1/{1.1\\Behoeften identificeren},
    1/2/{1.2\\Behoeften raadplegen en bevestigen},
    1/3/{1.3\\Uitgangsdoelen vaststellen},
    1/4/{1.4\\Concepten identificeren},
    1/5/{1.5\\Beschikbaarheid van gegevens controleren},
    1/6/{1.6\\Businesscase voorbereiden},
    2/1/{2.1\\Ontwerpen van outputs},
    2/2/{2.2\\Ontwerpen van variabelebeschrijvingen},
    2/3/{2.3\\Ontwerpen van verzameling},
    2/4/{2.4\\Ontwerpen van steekproefkader en steekproef},
    2/5/{2.5\\Ontwerpen van verwerking en analyse},
    2/6/{2.6\\Ontwerpen van productiesystemen en workflow},
    2/7/{2.7\\Ontwerpen van verspreiding},
    3/1/{3.1\\Verzamelinstrumenten bouwen},
    3/2/{3.2\\Verwerkings- en analyse-componenten bouwen of verbeteren},
    3/3/{3.3\\Verspreidings-componenten bouwen of verbeteren},
    3/4/{3.4\\Workflows configureren},
    3/5/{3.5\\Productiesysteem testen},
    3/6/{3.6\\Statistisch bedrijfsproces testen},
    4/1/{4.1\\Steekproefkader creëren en steekproef selecteren},
    4/2/{4.2\\Verzameling opzetten},
    4/3/{4.3\\Verzameling uitvoeren},
    4/4/{4.4\\Verzameling afronden},
    5/1/{5.1\\Gegevens integreren},
    5/2/{5.2\\Classificeren en coderen},
    5/3/{5.3\\Controleren, valideren en bewerken},
    5/4/{5.4\\Imputeren},
    5/5/{5.5\\Nieuwe variabelen en eenheden afleiden},
    5/6/{5.6\\Aggregaten berekenen},
    5/7/{5.7\\Gegevensbestanden afronden},
    6/1/{6.1\\Conceptuele outputs voorbereiden},
    6/2/{6.2\\Outputs valideren},
    6/3/{6.3\\Outputs interpreteren en uitleggen},
    6/4/{6.4\\Toepassen van onthullingscontrole},
    6/5/{6.5\\Outputs afronden},
    7/1/{7.1\\Outputsystemen bijwerken},
    7/2/{7.2\\Verspreidingsproducten produceren},
    7/3/{7.3\\Publicatie beheren},
    7/4/{7.4\\Verspreidingsproducten promoten},
    7/5/{7.5\\Gebruikers-ondersteuning beheren},
    8/1/{8.1\\Evaluatie-input verzamelen},
    8/2/{8.2\\Evaluatie uitvoeren},
    8/3/{8.3\\Actieplan overeenkomen}
    }{
    \node[contentbox] at (\x,\y) {};
    \node[contenttext] at (\x,\y) {\text};
    }	
%\foreach \x/\text in {
%	1/{1.\\Specify Needs},
%	2/{2.\\Design},
%	3/{3.\\Build},
%	4/{4.\\Collect},
%	5/{5.\\Process},
%	6/{6.\\Analyze},
%	7/{7.\\Disseminate},
%	8/{8.\\Evaluate}
%	}{
%	\node[titlebox] at (\x,0) {};
%	\node[titletext] at (\x,0) {\text};
%	}
%\foreach \x/\y/\text in {
%	1/1/{1.1\\Identify needs},
%	1/2/{1.2\\Consult and confirm needs},
%	1/3/{1.3\\Establish output objectives},
%	1/4/{1.4\\Identify concepts},
%	1/5/{1.5\\Check data availability},
%	1/6/{1.6\\Prepare business case},
%	2/1/{2.1\\Design outputs},
%	2/2/{2.2\\Design variable descriptions},
%	2/3/{2.3\\Design collection},
%	2/4/{2.4\\Design frame and sample},
%	2/5/{2.5\\Design processing and analysis},
%	2/6/{2.6\\Design production systems and workflow},
%	2/7/{2.7\\Design dissemination},
%	3/1/{3.1\\Build collection instruments},
%	3/2/{3.2\\Build or enhance processing and analysis components},
%	3/3/{3.3\\Build or enhance dissemination components},
%	3/4/{3.4\\Configure workflows},
%	3/5/{3.5\\Test production system},
%	3/6/{3.6\\Test statistical business process},
%	4/1/{4.1\\Create frame and select sample},
%	4/2/{4.2\\Set up collection},
%	4/3/{4.3\\Run collection},
%	4/4/{4.4\\Finalize collection},
%	5/1/{5.1\\Integrate data},
%	5/2/{5.2\\Classify and code},
%	5/3/{5.3\\Review, validate and edit},
%	5/4/{5.4\\Impute},
%	5/5/{5.5\\Derive new variables and units},
%	5/6/{5.6\\Calculate aggregates},
%	5/7/{5.7\\Finalize data files},
%	6/1/{6.1\\Prepare draft outputs},
%	6/2/{6.2\\Validate outputs},
%	6/3/{6.3\\Interpret and explain outputs},
%	6/4/{6.4\\Apply disclosure control},
%	6/5/{6.5\\Finalize outputs},
%	7/1/{7.1\\Update output systems},
%	7/2/{7.2\\Produce dissemination products},
%	7/3/{7.3\\Manage release},
%	7/4/{7.4\\Promote dissemination products},
%	7/5/{7.5\\Manage user support},
%	8/1/{8.1\\Gather evaluation inputs},
%	8/2/{8.2\\Conduct evaluation},
%	8/3/{8.3\\Agree an action plan}
%	}{
%	\node[contentbox] at (\x,\y) {};
%	\node[contenttext] at (\x,\y) {\text};
%	}
\end{tikzpicture}
\end{adjustwidth}




\end{frame}






\begin{frame}
\color{deepslateblue}\Large
\vfill
Wat is een statistiek?\\
\bigskip
Hoe statistieken beheren?\\
\bigskip
Hoe statistieken documenteren?\\
\bigskip
Hoe statistieken ontsluiten?\\
\vfill
\end{frame}
















\end{document}