% !TEX program = xelatex

\documentclass[00_handleiding_statistiekproductie.tex]{subfiles}



\begin{document}
	
	
\chapter*{Inleiding}




%Als we deze handleiding verspreiden binnen het netwerk zou ik in een voetnoot is toevoegen dat Statistiek Vlaanderen geen eengemaakte openbare statistiekdienst is maar het netwerk van VO-entiteiten die statistieken produceren en publiceren, maar dat we voor de leesbaarheid van deze tekst naar Statistiek Vlaanderen als een openbare statistiekdienst.



Één van de hoofdtaken van het netwerk Statistiek Vlaanderen en de Vlaamse Statistische Autoriteit is de productie en publicatie van openbare statistieken, ook wel Vlaamse Openbare Statistieken (VOS'en) genoemd. Maar wat verstaan we eigenlijk onder een statistiek en hoe produceren we die dan concreet? Om het netwerk goed te laten functioneren, is het cruciaal om met een gedeelde woordenschat te werken. Momenteel bestaan er echter grote verschillen in hoe termen en concepten binnen het netwerk worden gebruikt. Deze nota heeft daarom als doel enkele basisconcepten in de Vlaamse statistiekproductie te verduidelijken.

In deze handleiding wordt terminologie uit de GSIM-, GSBPM‐ en de SDMX‐standaarden gebruikt, al verwijzen we voorlopig slechts sporadisch in enkele kanttekeningen naar deze standaarden. De integratie van beide standaarden kan in een volgende versie opgenomen worden.


\end{document}