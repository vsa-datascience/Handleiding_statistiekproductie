% !TEX program = xelatex
% !BIB program = biber

%\PassOptionsToClass{Wordversion}{VSAreport} 
\documentclass[00_handleiding_statistiekproductie.tex]{subfiles}



\begin{document}
	
	
	
	
	
\chapter{Hoe bepalen we statistieken?}
\ref{ch:statistiekenbepalen}






\subsection{Officiële Statistieken}



% dit naar hoofdstuk 3?
We moeten hierbij ook opmerken dat het in de praktijk natuurlijk zo goed als onmogelijk is om alle gebruikersbehoeften in kaart te brengen, laat staan om aan al deze behoeften te voldoen. Dat is de reden waarom we als Statistiek Vlaanderen een officiële lijst aanleggen van statistieken die we produceren, namelijk de lijst van het Vlaams Statistisch Programma of VSP-lijst. Uiteraard is zo'n lijst nooit af en moet deze voortdurend geëvalueerd worden. Als Statistiek Vlaanderen hebben we dan ook de opdracht om de VSP-lijst elke legislatuur te herzien en dit doen we natuurlijk best door te blijven investeren in gebruikersonderzoek. Daarom moeten we voor elke statistiek in de VSP‐lijst ten minste volgende informatie voorzien:   
\begin{enumerate}[nosep]
	\item een overzicht van de gebruikersbehoeften,
	\item de conceptuele definitie inclusief een beschrijving van alle concepten
	\item de operationele definitie inclusief een operationalisering van alle concepten, en
	\item een kwaliteitsevaluatie van de conceptuele en operationele definitie in functie van de gebruikersbehoeften.
\end{enumerate}





% Misschien ook ergens iets invoegen over VOS, VOS op VSP-lijst en link met statistiekreeksen.
% vertrekkende vanuit het decreet
% doel van de tekst is ook voor een deel om duidelijk te maken wat een VOS precies is, dat moet dus ook duidelijk in de tekst staan.
% mss op het einde zelfs duidelijk zeggen een VOS is dus...



%De definitie van VOS’en volgens het Bestuursdecreet is erg ruim. Om de focus realistisch en haalbaar te houden, wordt de VOS-productie binnen Vlaanderen beperkt tot een afgebakende lijst bepaald in het VSP (de VSP-lijst). Deze lijst bevat de VOS’en die de SV-entiteiten of de Vlaamse regering relevant of noodzakelijk beschouwt voor het Vlaams beleid, of die moeten worden opgemaakt omwille van interfederale en internationale verplichtingen. Momenteel bevat de VSP-lijst ongeveer 350 statistieken maar er werd in de strategische blauwdruk afgesproken om de bestaande lijst in de komende jaren verder uit te breiden, rekening houdend met de beschikbare capaciteit binnen de entiteiten.



%https://ec.europa.eu/eurostat/statistics-explained/index.php?title=Beginners:Statistical_concept_-_What_are_official_statistics?
%"Official statistics are statistics produced within a national statistical system. National statistical systems include statistical organisations and units within a country that jointly collect, process and disseminate official statistics on behalf of the national government. Official statistics are usually collected within a legal framework, and in accordance with basic principles which ensure minimum professional standards, such as independency and objectivity. For example for the European Union, the legal framework is based on the European regulation (EC) No 223/2009 and the set of principles are called the European Statistics Code of Practice.
%Official statistics serve as a basis for decisions, for example, for politicians and policy makers: democratic societies cannot function properly without a solid basis of reliable and objective statistics.
%Non-official statistics include statistics collected and published by other bodies."




	Met Statistiek Vlaanderen leggen we op voorhand vast welke statistiekreeksen we publiceren. Deze statistiekreeksen zijn terug te vinden in het Vlaams Statistisch Programma (VSP‐lijst). Voor elke statistiek in de VSP‐lijst voorzien we ten minste volgende informatie:   
	% JO: Wat er in de VSP-lijst moet staan ligt vast door afspraken met IIS. Info die je voorstelt zou ik opnemen in SIMS-medata.
	\begin{enumerate}[nosep]
		\item de conceptuele definitie (die gebruikt wordt als handig label voor de statistiekreeks),
		\item het versienummer
		\item de operationele definitie inclusief een beschrijving van alle concepten (attributen en dimensies) inclusief de exacte waarden van die concepten,
		\item een overzicht van de gebruikersbehoeften, en
		\item een kwaliteitsevaluatie van de operationele definitie in functie van de gebruikersbehoeften.
	\end{enumerate}
	Tabel \ref{tab:voorbeeldVSPlijst} toont hoe de VSP-lijst er minimaal zou moeten uitzien. In de praktijk zullen we uiteraard informatie over statistiekreeksen niet bewaren in één gigantische tabel maar werken met informatiefiches per statistiekreeks gebaseerd op de   single integrated metadata structure (SIMS)\footnote{\href{https://ec.europa.eu/eurostat/web/metadata/reference-metadata-reporting-standards}{ec.europa.eu/eurostat/web/metadata/reference-metadata-reporting-standards}}.
	
%	\begin{table}
%		\caption{Het Vlaams Statistisch Programma (VSP) definieert een lijst van statistiekreeksen die Statistiek Vlaanderen publiceert. Elke statistiekreeks in deze reeks beschikt over een duidelijke conceptuele en operationele definitie, opgelijste gebruikersbehoeften en een kwaliteitsevaluatie.}
%		\label{tab:voorbeeldVSPlijst}
%		\begin{mdframed}
\newlist{reeks}{description}{1}
\setlist[reeks]{
	font=\bfseries\color{maincol1},
	labelindent=0pt,
	labelwidth=3ex,
	labelsep=1ex,
	itemindent=0pt,
	leftmargin=!,
	itemsep=2\baselineskip
	}
\newlist{features}{description}{1}
\setlist[features]{
	nosep,
	font=\mdseries\itshape\color{maincol2},
	labelindent=0pt,
	labelwidth=3ex,
	labelsep=1ex,
	itemindent=0pt,
	leftmargin=!
	}
\tiny
\begin{reeks}
\item[Aantal mannen en vrouwen, versie 1] \mbox{}
	\begin{features}
	\item[gebruikersvoorwaarden:] \mbox{}
		\begin{gebruikersvoorwaarden}
		\item Decreet XX.XX verwijst naar het aantal inwoners in Vlaamse gemeenten om beleid te voeren over \ldots
		\item Agentschap Binnenlands Bestuur publiceert cijfers over het aantal inwoners in gemeenten als \ldots
		\item Academische onderzoekers vragen cijfers over het aantal inwoners in gemeenten om onderzoek te voeren over \ldots
		\item \ldots 
		\end{gebruikersvoorwaarden}
	\item[Operationele definitie:]
	Het aantal en het percentage mannen en vrouwen in de feitelijke bevolking (aantal geregistreerde inwoners in het Rijksregister inclusief personen in het wachtregister en ambassadeurs) op 1 januari van elk kalenderjaar vanaf 2005 per Vlaamse gemeente volgens de NIS-code-indeling in 2019 = 600 cijfers per jaar.
	\item[Kwaliteit:]
	De gebruikers hebben eerder nood aan de grootte van de verblijvende bevolking in plaats van de feitelijke bevolking.
	\item[Nota:] Stopgezet in 2022, wegens aanpassing definitie.
	\end{features}
\item[Aantal mannen en vrouwen, versie 2] \mbox{}
	\begin{features}
	\item[gebruikersvoorwaarden:] \mbox{}
		\begin{gebruikersvoorwaarden}
		\item Decreet XX.XX verwijst naar het aantal inwoners in Vlaamse gemeenten om beleid te voeren over \ldots
		\item Agentschap Binnenlands Bestuur publiceert cijfers over het aantal inwoners in gemeenten als \ldots
		\item Academische onderzoekers vragen cijfers over het aantal inwoners in gemeenten om onderzoek te voeren over \ldots
		\item \ldots 
		\end{gebruikersvoorwaarden}
	\item[Operationele definitie:]
	Het aantal en het percentage mannen en vrouwen in de verblijvende bevolking (aantal geregistreerde inwoners in het Rijksregister inclusief personen in het wachtregister en ambassadeurs en personen die minder dan drie maanden in België verblijven) op 1 januari van elk kalenderjaar vanaf 2005 per Vlaamse gemeente volgens de NIS-code-indeling in 2019 = 600 cijfers per jaar. 
	\item[Kwaliteit:] Onderzoek XX toont aan dat er geen problemen zijn met deze statistiekreeks. 
	\item[Nota:] Stopgezet in 2025, wegens aanpassing gemeenten door fusies.
	\end{features}
\item[Aantal mannen en vrouwen, versie 3] \mbox{}
	\begin{features}
	\item[gebruikersvoorwaarden:] \mbox{}
		\begin{gebruikersvoorwaarden}
		\item Decreet XX.XX verwijst naar het aantal inwoners in Vlaamse gemeenten om beleid te voeren over \ldots
		\item Agentschap Binnenlands Bestuur publiceert cijfers over het aantal inwoners in gemeenten als \ldots
		\item Academische onderzoekers vragen cijfers over het aantal inwoners in gemeenten om onderzoek te voeren over \ldots
		\item \ldots 
		\end{gebruikersvoorwaarden}
	\item[Operationele definitie:]
	Het aantal en het percentage mannen en vrouwen in de verblijvende bevolking (aantal geregistreerde inwoners in het Rijksregister inclusief personen in het wachtregister en ambassadeurs en personen die minder dan drie maanden in België verblijven) op 1 januari van elk kalenderjaar vanaf 2005 per Vlaamse gemeente volgens de NIS-code-indeling in 2025 = 570 cijfers per jaar. 
	\item[Kwaliteit:]
	Onderzoek XX toont aan dat er geen problemen zijn met deze statistiekreeks.  
	\end{features}
\item[Tewerkstelling in hoogtechnologische sector, versie 1] \mbox{}
	\begin{features}
	\item[gebruikersvoorwaarden:] \mbox{}
		\begin{gebruikersvoorwaarden}
		\item Decreet XX.XX verwijst naar Tewerkstelling in hoogtechnologische sector in kader van \ldots
		\item De cijferpagina met cijfers over tewerkstelling in hoogtechnologische sector wordt X aantal keer per jaar geraadpleegd.
		\item \ldots
		\end{gebruikersvoorwaarden}	
	\item[Operationele definitie:] Percentage van de hele werkende bevolking aan de slag in de hoogtechnologische sector op 1 januari van elk kalenderjaar vanaf 2005 per Vlaamse gemeente en voor het hele VLaamse gewest volgens de NIS-code-indeling in 2019. De werkende bevolking omhelst \ldots De hoogtechnologische sector bestaat uit- bedrijven \ldots De data worden verzameld via de Enquête naar de Arbeidskrachten (EAK) door Statbel. In deze enquête wordt data verzameld door \ldots = 301 cijfers per jaar.
	\item[Kwaliteit:]
	Onderzoek YY toont aan dat er geen problemen zijn met deze statistiekreeks. 
	\end{features}
\item[Drinkwaterkwaliteit, versie 1] \mbox{}
	\begin{features}
	\item[gebruikersvoorwaarden:] \mbox{}
		\begin{gebruikersvoorwaarden}
		\item Decreet XX.XX verwijst naar drinkwaterkwaliteit in kader van \ldots
		\item \ldots
		\end{gebruikersvoorwaarden}
	\item[Operationele definitie:] Conformiteitspercentage van het kraantjeswater in heel Vlaanderen per jaar. Het conformiteitspercentage wordt berekend door Vlaamse Milieumaatschappij (VMM) op basis van het totale aantal analyses en het totale aantal vastgestelde normoverschrijdingen voor volgende parameters: \ldots = 1 cijfer per jaar.
	\item[Kwaliteit:]
	Volgens rapport ZZ ontstaan er kleine onzekerheidsfouten in de meting van parameter x, waardoor het werkelijke conformiteitspercentage kan afwijken met 0,2\%. Deze fluctuatie heeft slechts beperkte invloed op de kwaliteit waardoor geen herdefiniëring nodig is.
	\end{features}
\item[\ldots\ldots\ldots, versie \ldots] \mbox{}
	\begin{features}
	\item[gebruikersvoorwaarden:] \ldots
	\item[Operationele definitie:] \ldots
	\item[Kwaliteit:]\ldots	
	\end{features}
\item[\ldots\ldots\ldots]
\end{reeks}
\end{mdframed}
%	\end{table}
	



	Tabel \ref{tab:voorbeeldVSPlijst} toont hoe de VSP-lijst er minimaal zou moeten uitzien. In de praktijk zullen we uiteraard informatie over statistiekreeksen niet bewaren in één gigantische tabel maar werken met informatiefiches per statistiekreeks gebaseerd op de   single integrated metadata structure (SIMS)\footnote{\href{https://ec.europa.eu/eurostat/web/metadata/reference-metadata-reporting-standards}{ec.europa.eu/eurostat/web/metadata/reference-metadata-reporting-standards}}.





\section{Experimentele statistieken}

Om steeds beter en tijdig in te spelen op de behoeften van gebruikers is het voor een NSI zoals Statistiek Vlaanderen belangrijk steeds onderzoek te blijven uitvoeren naar nieuwe dataverzamelings- en -analysemethoden. Uit de toepassing van zo'n nieuwe methodes kunnen nieuwe cijfers of statistieken ontstaan die nog niet geofficialiseerd werden. De kwaliteit van zo'n statistieken is echter niet altijd meteen duidelijk en moet eerst diepgaander onderzocht worden. Dit soort statistieken worden daarom \emph{experimentele statistieken} genoemd \parencite{Eurostat2025Experimental}. 

Eerder in dit hoofdstuk zagen we al dat de kwaliteit van een statistiek wordt bepaald door de verhouding tussen gebruikersbehoeften en de comceptuele en operationele definities. Experimentele statistieken kunnen daarom in meerdere vormen voorkomen. Soms ontstaan ze spontaan vanuit nieuwe dataverzamelings- en analysemethoden maar is het niet duidelijk of ze relevant zijn omdat er geen duidelijke behoefte is aan de cijfers. Zo publiceren we als Statistiek Vlaanderen bijvoorbeeld cijfers over [voorbeeld toevoegen] 
maar weten we niet wie deze cijfers worden gebruikt en voor welk doeleind. Soms ontstaan experimentele statistieken ook als alternatief voor reeds bestaande officiële statistieken (VSP-lijst) met duidelijk geformuleerde gebruikersbehoeften. In dat geval rijst steeds de vraag of de nieuwe statistieken kwalitatief beter zijn dan de bestaande statistieken op het vlak van, onder andere, accuraatheid of tijdigheid. Zo onderzochten we bijvoorbeeld of de analyse van Twitter-berichten een alternatief kan vormen voor burgerbevragingen om sociaal sentiment in Vlaanderen te meten. Beide methodes hebben uiteraard hun voor- en nadelen en onderzoek moet uitwijzen welke methode de beste resultaten oplevert.

Wat precies onder ``nieuwe data‐verzamelings- en analysemethoden'' valt, is trouwens rekbaar. Vaak denken mensen bij dit begrip aan recente, innovatieve, complexe top notch data science technieken. Een experimentele statistiek kan echter ook worden ontwikkeld met behulp van oude en eenvoudige methoden. Zo kan een eenvoudige vraag in een bestaande bevraging al gelden als een nieuwe databron, en kunnen hieruit nieuwe cijfers worden berekend en gepubliceerd waarvoor de gebruikersbehoeften nog niet duidelijk in kaart werden gebracht. Ook in deze situatie spreken we van een experimentele statistiek.

Bij experimentele statistieken is het belangrijk om steeds op voorhand een duidelijk evaluatiemoment in te plannen, bijvoorbeeld vijf jaar na de eerste publicatie. Tijdens dit evaluatieonderzoek worden onder andere volgende vragen gesteld:
\begin{itemize}[nosep]
    \item Wordt de statistiek effectief gebruikt, en zo ja, waarvoor?
    \item Komt het gebruik en de interpretatie overeen met de operationele definitie?
    \item Zijn er suggesties van gebruikers om de statistiek beter af te stemmen op hun noden?
\end{itemize}
Op basis van de resultaten van dit onderzoek kan de statistiek:
\begin{itemize}[nosep]
    \item worden geüpgraded tot een officiële statistiek;
    \item worden aangepast om beter aan te sluiten bij gebruikersbehoeften;
    \item worden stopgezet indien de kwaliteit onvoldoende blijkt.
\end{itemize}


	% Suggestie voor verdere uitwerking:
	% Hier kan eventueel een aparte sectie volgen over de werking rond onderzoek en innovatie binnen de VSA, met taken zoals organisatie van redactieraden, seminaries, data science-projecten en academische samenwerking.	
	
	
	%Alle onderzoeks- en innovatieprojecten van de VSA worden geïntegreerd in een gecentraliseerde werking. Het gaat hierbij om zowel inhoudelijk verdiepend onderzoek, beleidsdomeinoverschrijdend onderzoek, onderzoek rond statistiekontwikkeling, onderzoek in het kader van academische samenwerking of data science-projecten. De activiteiten van deze groep zijn wel steeds gericht op de (verkenning van de) ontwikkeling van nieuwe openbare statistieken of de verbetering van bestaande statistieken. Daarnaast is er binnen deze werking ook plaats voor onderzoek naar efficiëntere manieren om de opdrachten en interne processen van de VSA uit te voeren (bv. experimenten met nieuwe analyse- of visualisatiesoftware). 
	%De werking rond onderzoek en innovatie neemt volgende taken voor haar rekening:
	%•	Organisatie van de redactieraad. Bij voorkeur mondt elk onderzoeks- of innovatieproject minstens uit in een rapport.
	%•	Organisatie van (interne) seminaries waarbij onderzoek wordt voorgesteld aan de collega’s en andere geïnteresseerden.
	%•	Organisatie van de data science-projecten
	%•	Organisatie van de academische samenwerking zoals de ondersteuning van thesisstudenten.
	
















\subsubsection{Hoe bepalen we concepten en dimensies?}

\begin{info}
	We hebben hier een hiaat in het SDMX model blootgelegd. De documentatie van SDMX bespreekt nergens hoe attributen dimensies kunnen worden of vice versa wanneer datasets met statistiekreeksen worden samengevoegd of uitgesplitst. Om deze reden is de uitwerking van attributen in het model ook veel beperkter dan de uitwerking van dimensies. Hierdoor kan SDMX moeilijker gelinkt worden aan de ideeën rond conceptuele definities, operationele definities en kwaliteit van statistiekreeksen.
	
	Sterker nog, binnen SDMX wordt parameter nooit duidelijk beschouwd als een dimensie waardoor je verschillende parameters zoals bevolkingsgrootte en oppervlakte moeilijk in één tabel kan combineren zonder in problemen te komen over de definitie van dimensies en attributen.
\end{info}



\begin{info}
Binnen de huidige SDMX-standaard stellen we statistiekreeksen voor met slechts één kolom voor de cijfers. De structuur in tabel \ref{tab:voorbeeldcijferreeks} zou hiervoor gepivoteerd moeten worden. Hoogstwaarschijnlijk versoepelt de SDMX-standaard echter in de toekomst waardoor verschillende kolommen cijfers kunnen bevatten. 
\end{info}


De bepaling van statistiekreeksen op basis van concepten is geen exacte wetenschap, maar eerder een arbitrair proces waarin verschillende keuzes gemaakt moeten worden. Een eerste keuze is welke concepten je gebruikt als attributen om statistiekreeksen te onderscheiden en welke als dimensies binnen deze statistiekreeksen. In het ene uiterste gebruik je alle concepten als attributen waardoor een aparte reeks voor elke individuele statistiek ontstaat, zoals in tabel \ref{tab:reeksapart}. In het andere uiterste gebruik je alle concepten als dimensies waardoor één enkele reeks ontstaat met alle cijfers erin, zoals in tabel \ref{tab:reekssamen}. Beide benaderingen zijn uiteraard onpraktisch. We moeten op zoek naar een evenwicht dat zowel overzichtelijk als werkbaar is.

\begin{table}
	\caption{Concepten kunnen worden gebruikt om aparte cijferreeksen te definiëren of als dimensie in één grote cijferreeks. Geen van beide situaties is echter werkbaar. We moeten een middenweg vinden.}
	\label{tab:reekscombineren}
	\footnotesize
\begin{subtable}{0.4\linewidth}
\caption{Elk cijfer in aparte statistiekreeks.}
\label{tab:reeksapart}
\begin{tblr}{
	width=\linewidth,
	colspec={X[l]},
	row{1,4,7,10,13} = {fg=white,bg=maincol1},	
	row{2,5,8,11,14} = {fg=textcol,bg=white},
	row{3,6,9,12,15} = {rowsep=-2pt},
	vline{1,Z} = {1-2,4-5,7-8,10-11,14-15}{maincol1},
	hline{3,6,9,12,15} = {maincol1}	
	}
Aantal inwoners in Vlaanderen \\
6\,821\,770 \\\\
Percentage Vlamingen met gevorderde digitale vaardigheden \\
26,1 \\\\
Aantal zeugen in Vlaanderen \\
339\,103 \\\\
Ondergrens 95\%-CI gemiddelde vertrouwen in provinciale overheid \\
2,34 \\\\
\ldots \\
\ldots \\
\end{tblr}
\end{subtable}
\hfill
\begin{subtable}{0.5\linewidth}
\caption{Één statistiekreeks voor alle cijfers.}
\label{tab:reekssamen}
\begin{tblr}{
	width=\linewidth,
	colspec={X[l]Q[c]},
	rows = {valign=m},
	column{1} = {fg=white,bg=maincol2},
	row{1} = {fg=white,bg=maincol1},
	hlines = {maincol1},
	vline{1,Z} = {maincol1}
	}
Parameter & Cijfer \\
Aantal inwoners in Vlaanderen & 6\,821\,770 \\
Percentage Vlamingen met gevorderde digitale vaardigheden & 26,1 \\
Aantal zeugen in Vlaanderen & 339\,103 \\
Ondergrens 95\%-CI gemiddelde vertrouwen in provinciale overheid & 2,34 \\
\ldots & \ldots \\
\end{tblr}
\end{subtable}
\end{table}

De tweede keuze is welke concepten binnen een statistiekreeks worden samengevoegd en welke worden uitgesplitst. Bijvoorbeeld: cijfers over mannen en vrouwen in gemeenten kun je uitsplitsen op basis van ‘parameter’ en ‘geslacht’, zoals in tabel \ref{tab:dimensieapart}. Je kunt er echter ook voor kiezen om beide concepten te combineren in één dimensie, zoals in tabel \ref{tab:dimensiesamen}. Ook hier gaat het om een afweging tussen eenvoud en gebruiksgemak.

\begin{table}
	\caption{De indeling van dimensies is arbitrair. Kenmerken kunnen gecombineerd worden in één dimensie of als aparte dimensies worden opgenomen.}
	\label{tab:dimensieskiezen}
	\tiny 
\begin{subtable}{0.45\textwidth}
\caption{Geslacht en parameter worden gecombineerd als één dimensie.}
\label{tab:dimensiesamen}
\begin{tblr}{
	width=\linewidth,
	colspec={cX[c]c},
	column{1-2} = {fg=textcol,bg=maincol3},
	row{1} = {fg=white,bg=maincol1},
	row{2} = {fg=white,bg=maincol2},
	hline{Z} = {maincol1},
	vline{1,Z} = {maincol1}
	}
\SetCell[c=3]{l} Bevolkingsaantal 2019 \\
Gemeente & Parameter & Observatie \\
A & totaal aantal & 300 \\
A & aantal vrouwen & 180 \\
A & percentage vrouwen & 60 \\
A & aantal mannen & 120 \\
A & percentage mannen & 40 \\
B & totaal aantal & 500 \\
B & aantal vrouwen & 100 \\
B & percentage vrouwen & 20 \\
B & aantal mannen & 400 \\
B & percentage mannen & 80 \\
\end{tblr}
\end{subtable}
\hfill
\begin{subtable}{0.45\textwidth}
\caption{Geslacht en parameter zijn aparte dimensies.}
\label{tab:dimensieapart}
\begin{tblr}{
	width=\linewidth,
	colspec={cX[c]X[c]c},	
	column{1-3} = {fg=textcol,bg=maincol3},
	row{1} = {fg=white,bg=maincol1},
	row{2} = {fg=white,bg=maincol2},
	hline{Z} = {maincol1},
	vline{1,Z} = {maincol1}
	}
\SetCell[c=3]{l} Bevolkingsaantal 2019 \\
Gemeente & Geslacht & Parameter & Observatie \\
A & totaal  & aantal     & 300 \\
A & vrouwen & aantal     & 180 \\
A & vrouwen & percentage & 60  \\
A & mannen  & aantal     & 120 \\
A & mannen  & percentage & 40  \\
B & totaal  & aantal     & 500 \\
B & vrouwen & aantal     & 100 \\
B & vrouwen & percentage & 20  \\
B & mannen  & aantal     & 400 \\
B & mannen  & percentage & 80  \\
\end{tblr}
\end{subtable}
\end{table}

Een derde keuze is de definitie van de concepten zelf. Stel dat je cijfers publiceert over inwoners van Vlaamse gemeenten, uitgesplitst naar geslacht en leeftijd, zonder deze kenmerken te kruisen. Je kunt hiervoor aparte dimensies definiëren voor ‘geslacht’ en ‘leeftijd’, zoals in tabel \ref{tab:dimensiegesplitst}. Je kan deze dimensies echter ook herdefiniëren tot één dimensie ‘bevolkingsgroep’, die alle waarden van geslacht en leeftijd bevat, zoals in tabel \ref{tab:dimensiegecombineerd}.

\begin{table}
	\centering\tiny
	\caption{Concepten in een statistiekreeks kunnen arbitrair worden geherdefinieerd.}
	\label{tab:dimensiedef}
	\tiny 
\begin{subtable}{0.45\textwidth}
\caption{Aparte dimensies voor geslacht en leeftijd.}
\label{tab:dimensiegesplitst}
\begin{tblr}{
	width=\linewidth,
	colspec={cX[c]X[c]c},
	column{1-3} = {fg=textcol,bg=maincol3},
	row{1} = {fg=white,bg=maincol1},
	row{2} = {fg=white,bg=maincol2},
	hline{Z} = {maincol1},
	vline{1,Z} = {maincol1}
	}
\SetCell[c=3]{l} Bevolkingsaantal 2019 \\
Gemeente & Geslacht & Leeftijd & Observatie \\
A & mannen  & totaal     & 90 \\
A & vrouwen & totaal     & 130 \\
A & totaal  & 18-30 jaar & 60 \\
A & totaal  & 31-65 jaar & 120 \\
A & totaal  & 66+ jaar   & 40 \\
B & mannen  & totaal     & 350 \\
B & vrouwen & totaal     & 150 \\
B & totaal  & 18-30 jaar & 20 \\
B & totaal  & 31-65 jaar & 400 \\
B & totaal  & 66+ jaar   & 80 \\
\end{tblr}
\end{subtable}
\hfill
\begin{subtable}{0.45\textwidth}
\caption{Geslacht en leeftijd worden gecombineerd in de dimensie `bevolkingsgroep'.}
\label{tab:dimensiegecombineerd}
\begin{tblr}{
	width=\linewidth,
	colspec={cX[c]c},
	column{1-2} = {fg=textcol,bg=maincol3},
	row{1} = {fg=white,bg=maincol1},
	row{2} = {fg=white,bg=maincol2},
	hline{Z} = {maincol1},
	vline{1,Z} = {maincol1}
	}
\SetCell[c=3]{l} Bevolkingsaantal 2019 \\
Gemeente & Bevolkingsgroep & Observatie \\
A & mannen  & 90 \\
A & vrouwen & 130 \\
A & 18-30 jarigen & 60 \\
A & 31-65 jarigen & 120 \\
A & 66+ jarigen   & 40 \\
B & mannen   & 350 \\
B & vrouwen & 150 \\
B & 18-30 jarigen & 20 \\
B & 31-65 jarigen & 400 \\
B & 66+ jarigen   & 80 \\
\end{tblr}
\end{subtable}
\end{table}

Deze voorbeelden tonen aan dat de bepaling van statistiekreeksen arbitrair is. Richtlijnen zijn daarom essentieel om consistentie en werkbaarheid te waarborgen.

\begin{richtlijnen}
	
	% Hier ook zeker kijken wat werkt voor Lieven en Georneth. kijken hoe ze het nu doen. Georneth en Lieven moeten betrokken worden bij datamodel, kunnen hier ook verantwoordelijkheid nemen.
	
	\item Cijfers die enkel verschillen in geografische indeling worden gecombineerd in één statistiekreeks waarbij deze geografische indeling als dimensie wordt opgevat. Bij voorkeur wordt voor zo’n geografische dimensie de NIS‐code‐indeling van Statbel gebruikt waarbij ook de versie van de NIS‐code‐indeling vermeld wordt. Naast NIS‐codes kunnen echter ook andere geografische gebieden, zoals postcodes, toeristische regio’s of ziekenhuisnetwerken, worden opgenomen. Voor hiërarchische indelingen kunnen voor de volledigheid extra dimensies worden toegevoegd zoals in tabel \ref{tab:hierdim}.  
	
	\begin{table}
		\caption{Voor hiërarchische dimensies kunnen bijkomende dimensies worden toegevoegd, hoewel niet strikt noodzakelijk.}
		\label{tab:hierdim}
		\scriptsize
\begin{tblr}{
	width=\linewidth,
	colspec={X[l]X[l]X[l]c},
	column{1-3} = {fg=textcol,bg=maincol3},
	row{1} = {fg=white,bg=maincol1},
	row{2} = {fg=white,bg=maincol2},
	hline{Z} = {maincol1},
	vline{1,Z} = {maincol1}
	}
\SetCell[c=3]{l} Bevolkingsaantal 2019 \\	
Geografisch gebied & Gewest & Provincie & Aantal inwoners \\
Vlaams Gewest &  &  & 6\,589\,069 \\
Waals Gewest &  &  & 3\,633\,795 \\
Brussels Gewest &  &  & 1\,208\,542 \\
Prov. Antwerpen & Vlaams Gewest &  &  1\,857\,986 \\
Aartselaar & Vlaams Gewest & Prov. Antwerpen & 14\,293 \\
Antwerpen & Vlaams Gewest & Prov. Antwerpen & 525\,935 \\
Boechout & Vlaams Gewest & Prov. Antwerpen & 13\,266 \\
Boom & Vlaams Gewest & Prov. Antwerpen & 18\,244 \\
Borsbeek & Vlaams Gewest & Prov. Antwerpen & 10\,854 \\
\ldots & \ldots & \ldots & \ldots \\
\end{tblr}
	\end{table}
	
	\item Cijfers die alleen verschillen in tijdsperiode worden samengevoegd in één statistiekreeks, met de tijdsperiode als dimensie.  
	
	\begin{info}
		De keuze om tijdsperiodes als dimensie te behandelen, betekent niet dat alle cijfers steeds in één dataset opgeslagen moeten worden. Zo is het op een file server handiger datasets per jaar te bewaren, terwijl deze datasets toch statistieken bevatten uit dezelfde statistiekreeks.
	\end{info}
	
	\item Cijfers die op dezelfde manier worden berekend en uit dezelfde bron worden afgeleid, worden zoveel mogelijk gecombineerd in één reeks. Dit vereenvoudigt de verwerking en documentatie. 
	
	\item % schrappen?  als er teveel uitzonderingen zijn, is het dan nodig deze te houden?  beetje onduidelijk
	Statistische parameters zoals frequenties en gemiddelden  worden zo veel mogelijk opgesplitst in aparte reeksen. Dit vergemakkerlijkt de beschrijving van de statistiekreeks. Parameters die op een natuurlijke manier bij elkaar horen, vormen een uitzondering en worden wel zo veel mogelijk gecombineerd in dezelfde statistiekreeks: 
	\begin{itemize}
		\item Proporties/percentages berekend op frequenties uit één enkele statistiekreeks, worden ook aan deze statistiekreeks toegevoegd. 
		\item Inferentiële statistieken zoals de onder- en bovengrens van betrouwbaarheidsintervallen, standaardfouten of $p$-waarden worden toegevoegd aan de statistiekreeks met de bijhorende puntschattingen.
	\end{itemize} 
	
	\begin{info}
		Percentages kunnen op verschillende manieren worden gedefinieerd. Dit moet altijd duidelijk worden omschreven in de operationele definitie en de parameterdimensie (bijv. het percentage mannen in gemeente X is niet hetzelfde als het percentage Vlaamse mannen die in gemeente X wonen).
	\end{info}
	
	\item Binnen een statistiekreeks worden de verschillende dimensies zo goed mogelijk volledig met elkaar gekruist. Als dit niet mogelijk of wenselijk is, kunnen dimensies worden samengevoegd, bijvoorbeeld een dimensie ``bevolkingsgroep'' in plaats van aparte dimensies ``geslacht'', ``leeftijd'' en ``nationaliteit'' als deze dimensies niet worden gekruist.  
	
	\item Cijfers die gebaseerd zijn op meerdere basisreeksen afgeleid uit verschillende databronnen, zoals bevolkingsdichtheid (gebaseerd op inwonersaantallen en geografische oppervlakten), worden altijd in aparte reeksen ondergebracht. Dit bevordert overzichtelijkheid en documentatie.
	% Een cijfers dat....  wordt opgesplitst in verschillende reeksen.  Dus dat teller en noemers bijvoorbeeld aparte reeksen zijn als het uit verschillende bronnen komt.
	
	\item Alle concepten worden verzameld in een centrale conceptenlijst dat dient al datamodel. Dit datamodel wordt beheerd door een toegewezen team van collega's. Bij elk concept wordt ook een codelijst voorzien die alle mogelijke waarden op het concept bevat.  
	% conceptenlijst is dus een datamodel
	
	
\end{richtlijnen}








\subsubsection{Versies van statistiekreeksen}

Statistiekreeksen kunnen doorheen de tijd veranderen om verschillende redenen:
\begin{itemize}
	\item De operationele definitie van een statistiekreeks wordt aangepast omdat dimensies veranderen, bijvoorbeeld de gemeenteindeling verandert door gemeentefusies.
	\item De operationele definitie wordt aangepast in lijn met nieuwe gebruikersbehoeften.   
	\item Er werd een fout ontdekt in de berekening van cijfers en deze fout wordt gecorrigeerd.
\end{itemize}
Omdat we verwacht worden elk gepubliceerd cijfer beschikbaar te houden, creëren we in elk van bovenstaande situaties een nieuwe versie van de statistiekreeks. Deze versie duiden we aan door een versienummer terwijl de cijfers en documentatie van de oude versie bewaard blijven. In het geval van een berekeningsfout voegen we een disclaimer toe aan de documentatie van de oude versie met uitleg over deze fout. Bij wijzigende definities, documenteren we eveneens waarom deze wijziging nodig was. Bij een nieuwe versie van een statistiekreeks publiceren we cijferreeksen ook retrospectief indien dat mogelijk is en gewenst wordt geacht.














\end{document}