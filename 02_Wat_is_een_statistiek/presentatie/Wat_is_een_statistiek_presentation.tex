% !TEX program = xelatex

\documentclass{beamer}
\usetheme{VSA}

\usepackage{tikz}
\usepackage{tabularx}
\newcolumntype{L}{>{\raggedright\arraybackslash}X}
\usepackage{tabularray}
\usepackage[framemethod=TikZ]{mdframed}

\newcommand\point[1]{{\bfseries\color{maincol2}#1\par}}
\newcommand\explanation[1]{{\footnotesize\quad#1\par}}



\begin{document}


\title{Wat is een statistiek?}
\author{Jorre Vannieuwenhuyze}
\date{}
%\date{13 oktober 2025}
%\institute{Stuurgroep transitietraject}

\titleframe


\begin{frame}
\begin{quote}
\makebox[0pt][r]{``}Statistiek Vlaanderen is het netwerk van Vlaamse overheidsinstanties die \textcolor{maincol2}{openbare statistieken} ontwikkelen, produceren en publiceren.''
\end{quote}
\end{frame}



\begin{frame}
\color{maincol2}\bfseries\Large
\vfill
Wat is een statistiek?\\
\bigskip
Hoe statistieken documenteren?\\
\bigskip
Hoe statistieken beheren?\\
\bigskip
Hoe statistieken ontsluiten?\\
\bigskip
Hoe statistieken afbakenen?
\vfill
\end{frame}






\tableofcontentsframe



\section{Statistiek als een cijfer}


\begin{frame}
\frametitle{\onslide<2->{Een statistiek bestaat uit\\een cijfer en attributen.}}
\centering
\onslide<2->{
	{\Large\bfseries attributen}\par
	(betekenis/metadata)
	\vspace{\baselineskip}\par
	\faArrowDown\par
	\vspace{\baselineskip}\par
	}
\textcolor{maincol2}{\Large\bfseries 525\,935}\par
\end{frame}





\begin{frame}
\frametitle{Attribuut 1 is de conceptuele definitie.}
\framesubtitle{Wat betekent de statistiek?}
\begin{center}
\onslide<2->{
	het aantal inwoners in Antwerpen in 2019
	\vspace{\baselineskip}\par
	\faArrowDown\par
	\vspace{\baselineskip}\par
	}
\textcolor{maincol2}{\Large\bfseries 525\,935}\\
\end{center}
\end{frame}





\begin{frame}
\frametitle{Een conceptuele definitie bestaat uit\\verschillende concepten}
\hfill
\begin{tikzpicture}[
	values/.style={anchor=west,text height=1.5ex,text depth=0pt,inner sep=1pt,xshift=.5ex},
	line/.style={maincol2,line width=.5pt},
	concept/.style={maincol2,font=\small,anchor=south,inner sep=1pt}
	]
\node[values] (A) {het aantal inwoners};
\node[values] (B) at (A.east) {in Antwerpen};
\node[values] (C) at (B.east) {in 2019};
\draw[line] (A.north west) -- (A.north east) (A.north) -- +(0,20pt) node[concept]{parameter};
\draw[line] (B.north west) -- (B.north east) (B.north) -- +(0,20pt) node[concept]{gemeente};
\draw[line] (C.north west) -- (C.north east) (C.north) -- +(0,20pt) node[concept]{jaartal};
\onslide<2->{
	\node[values,xshift=-5mm,anchor=east] (Z) at (A.west) {waarden:};
	\node[concept,anchor=south east,yshift=20pt] at (Z.north east) {concepten:};
	}
\end{tikzpicture}
\end{frame}

\begin{frame}
\frametitle{\onslide<2->{De afbakening van concepten is arbitrair.}}
\vfill
\begin{small}\color{maincol2} 
\textbf{Oefening:} Wat zijn de concepten in volgende conceptuele definitie?
\end{small}
\par
\smallskip
\par 
\begin{scriptsize} 
\makebox[0pt][r]{``}het aantal inwoners in Antwerpen per bevolkingsgroep (mannen, vrouwen, 0- tot 18-jarigen, 19- tot 64- jarigen, 65+'ers, mensen met Belgische nationaliteit, mensen met buitenlandse nationaliteit,\ldots'')
\end{scriptsize}
\par
\vfill
\par
\onslide<2->{
	\begin{small}\color{maincol2}
	\textbf{Antwoord:}
	\end{small}
	\par
	\smallskip
	\par
	\begin{itemize}\scriptsize
	\item parameter, gemeente, geslacht, leeftijd, nationaliteit,\ldots\ of
	\item parameter, gemeente, bevolkingsgroep
	\end{itemize}
	}
\end{frame}



\begin{frame}
\frametitle{Attribuut 2 is de gebruikersbehoefte.}
\framesubtitle{Waarom produceren we de statistiek?}
\vfill
\onslide<2->{
	\begin{itemize}\footnotesize
	\item Decreet XX.XX verwijst naar het aantal inwoners in Vlaamse gemeenten in de context van \ldots\\
	\item Agentschap Binnenlands Bestuur publiceert cijfers over het aantal inwoners in gemeenten om beleid te ondersteunen rond \ldots\\
	\item Academische onderzoekers vragen cijfers over het aantal inwoners in gemeenten om onderzoek te voeren over \ldots\\
	\end{itemize}
	\begin{center}
	\LARGE$\Downarrow$\\
	\end{center}
	}
\begin{center}
\textcolor{maincol2}{\Large\bfseries 525\,935}\\
{\footnotesize het aantal inwoners in Antwerpen in 2019}
\end{center}
\end{frame}





\begin{frame}
\frametitle{Attribuut 3 is de operationele definitie.}
\framesubtitle{Hoe wordt de statistiek precies berekend?}
\pause
\tiny
\makebox[0pt][r]{``}De grootte van de wettelijke bevolking op 1 januari 2019, 0.00 uur, van de gemeente Antwerpen (NIS 11002 in 2019). De grootte van de wettelijke bevolking wordt gedefinieerd en aangeleverd door Statbel op basis van het Rijksregister van de natuurlijke personen. Dit Rijksregister bevat, onder andere, het bevolkingsregister en het vreemdelingenregister die de wettelijke bevolking bepalen. Het bevolkingsregister bevat alle Belgen en buitenlanders die gemachtigd zijn tot vestiging op het Belgisch grondgebied, en het vreemdelingenregister bevat alle buitenlanders die toegelaten of gemachtigd zijn tot een verblijf van meer dan 3 maanden op het Belgisch grondgebied, hetzij voor bepaalde of onbepaalde duur.\par	
Bepaalde categorieën buitenlanders (vb. diplomatiek en consulair personeel) zijn vrijgesteld van inschrijving in de bevolkingsregisters en worden daardoor niet meegerekend bij de wettelijke bevolking. In sommige gevallen kunnen zij op eigen vraag wel ingeschreven worden. Enkel in dat geval worden zij meegerekend in de bevolkingscijfers.\par 		
Het Rijksregister omvat verder ook een wachtregister voor asielzoekers en een wachtregister voor EU‐burgers. Het wachtregister voor asielzoekers bevat alle verzoekers om internationale bescherming die worden ingeschreven door de Dienst Vreemdelingenzaken (DVZ). In 1995 besliste Statbel de personen in dit wachtregister niet meer mee te tellen bij de wettelijke bevolking. Pas nadat asielzoekers worden overgeschreven van het wachtregister naar het bevolkingsregister of het vreemdelingenregister, worden zij opgenomen in de bevolkingsstatistieken van Statbel. Zo'n overschrijving naar het bevolkingsregister of het vreemdelingenregister gebeurt na erkenning als vluchteling, na toekenning van een statuut subsidiaire bescherming, of na verwerving van een verblijfsvergunning om een andere reden.\par		
Verder bevat het Rijksregister ook een wachtregister voor EU‐burgers in afwachting van woonstcontrole. Deze personen worden evenmin meegeteld bij de wettelijke bevolking. Pas na woonstcontrole worden deze personen overgeschreven naar het vreemdelingenregister en worden zij meegeteld in de wettelijke bevolking.''
\end{frame}



\begin{frame}
\frametitle{De operationele definitie vertaalt \\elk concept van abstract naar concreet.}
\framesubtitle{\onslide<5->{Operationaliseren is keuzes maken.}}
\scriptsize
\begin{tblr}{
	colspec={llcX[l]cX[l]cc},
	column{1} = {bg=maincol2,fg=white},
%	column{2-4} = {colsep=2pt},
	row{1,2} = {bg=maincol1,fg=white},
	column{5,7} = {font=\color{palgray4}},
	hline{2} = {white},
	hline{4,5} = {maincol1},
	hline{Z} = {maincol1},
	vline{1,Z} = {maincol1},
	vline{2} ={white},
	}
& Conceptueel &$\rightarrow$& Operationeel  \\		
& Abstract &$\rightarrow$& Concreet \\
parameter & aantal inwoners & \onslide<2->{$\rightarrow$} &  \onslide<2->{aantal inwoners volgens de wettelijke bevolking} & \onslide<5->{of} & \onslide<5->{aantal inwoners volgens de verblijvende bevolking} & \onslide<5->{of} & \onslide<5->{\ldots} \\
gemeente & in Antwerpen & \onslide<3->{$\rightarrow$} & \onslide<3->{in Antwerpen volgens NIS-code 11002 in 2019} & \onslide<5->{of} & \onslide<5->{in Antwerpen volgens NIS-code 11002 in 2025} & \onslide<5->{of} & \onslide<5->{\ldots} \\
jaartal & in 2019 & \onslide<4->{$\rightarrow$} & \onslide<4->{op 1 januari 2019, 0.00 uur} & \onslide<5->{of} & \onslide<5->{op 1 juli 2019, 0.00 uur} & \onslide<5->{of} & \onslide<5->{\ldots} \\ 
\end{tblr}
\end{frame}




\begin{frame}
\frametitle{Attribuut 4 is een kwaliteitsevaluatie.}
\framesubtitle{Hoe goed beantwoordt de statistiek gebruikersbehoeften?}
\vfill
\point{Relevantie}
\explanation{Beantwoordt elke statistiek een behoefte?}
\explanation{Worden alle behoeften beantwoord?}
\vfill
\pause
\point{Nauwkeurigheid}
\explanation{Meet de statistiek wat het moet meten?}
\explanation{Is de statistiek geldig en betrouwbaar?}
\vfill
\pause
\point{Tijdigheid \& Punctualiteit}
\explanation{Wordt de statistiek snel gepubliceerd?}
\explanation{Wordt de statistiek op een duidelijk moment gepubliceerd?}
\vfill
\pause
\point{Vergelijkbaarheid}
\explanation{Kan je de statistiek vergelijken met andere cijfers?}
\vfill
\pause
\point{Legaliteit}
\explanation{Is de statistiek berekend conform wettelijke voorschriften (bv. GDPR)?}
\vfill
\end{frame}



\begin{frame}
\frametitle{Andere attributen zijn ook mogelijk.}
\vfill
\point{Publiceerbaarheid}
\explanation{Kan de statistiek publiek gepubliceerd worden of enkel intern?}
\vfill
\point{Eigenaarschap}
\explanation{Wie draagt eindverantwoordelijkheid en is aanspreekbaar?}
\vfill
\point{Contactinformatie}
\explanation{Wie kan je contacteren omtrent de statistiek?}
\vfill
\point{\ldots}
\explanation{\ldots}
\vfill
\end{frame}







\section{Statistiek als een cijfertabel}


\begin{frame}
\frametitle{Een cijfertabel combineert cijfers\\met overlappende concepten.}
\begin{center}
\tiny
\begin{tblr}{
	width=.8\linewidth,
	colspec={X[l]X[c]},
	column{1} = {fg=textcol,bg=maincol3},
	row{1} = {fg=white,bg=maincol2},
	hline{1,Z} = {maincol1},
	vline{1,Z} = {maincol1},
	hspan=minimal
	}
Conceptuele definitie & Cijfer \\
Aantal inwoners in Aalst in 2019 & 86\,445 \\
Aantal inwoners in Aalter in 2019 & 28\,906 \\
Aantal inwoners in Aarschot in 2019 & 30\,115 \\
\ldots & \ldots \\
Aantal inwoners in Zwijndrecht in 2019 & 19\,056 \\
\end{tblr}
\end{center}
\vfill\mbox{}\\
Concept 1: Parameter\\
Concept 2: Gemeente\\
Concept 3: Jaartal\\
\end{frame}


\begin{frame}
\frametitle{Concepten met variërende waarden\\noemen we dimensies.\strut}
\begin{center}
\tiny
\begin{tblr}{
	width=.8\linewidth,
	colspec={X[l]X[c]},
	column{1} = {fg=textcol,bg=maincol3},
	row{1} = {fg=white,bg=maincol2},
	hline{1,Z} = {maincol1},
	vline{1,Z} = {maincol1},
	hspan=minimal
	}
Conceptuele definitie & Cijfer \\
Aantal inwoners in Aalst in 2019 & 86\,445 \\
Aantal inwoners in Aalter in 2019 & 28\,906 \\
Aantal inwoners in Aarschot in 2019 & 30\,115 \\
\ldots & \ldots \\
Aantal inwoners in Zwijndrecht in 2019 & 19\,056 \\
\end{tblr}
\end{center}
\vfill\mbox{}\\
Concept 1: Parameter\\
\textbf{Concept 2: Gemeente \textcolor{maincol2}{$\rightarrow$ dimensie}}\\
Concept 3: Jaartal\\
\end{frame}


\begin{frame}
\frametitle{Concepten die constant blijven\\noemen we attributen.\strut}
\begin{center}
\tiny
\begin{tblr}{
	width=.8\linewidth,
	colspec={X[l]X[c]},
	column{1} = {fg=textcol,bg=maincol3},
	row{1} = {fg=white,bg=maincol2},
	hline{1,Z} = {maincol1},
	vline{1,Z} = {maincol1},
	hspan=minimal
	}
Conceptuele definitie & Cijfer \\
Aantal inwoners in Aalst in 2019 & 86\,445 \\
Aantal inwoners in Aalter in 2019 & 28\,906 \\
Aantal inwoners in Aarschot in 2019 & 30\,115 \\
\ldots & \ldots \\
Aantal inwoners in Zwijndrecht in 2019 & 19\,056 \\
\end{tblr}
\end{center}
\vfill\mbox{}\\
\textbf{Concept 1: Parameter \textcolor{maincol2}{$\rightarrow$ attribuut}}\\
Concept 2: Gemeente\\
\textbf{Concept 3: Jaartal \textcolor{maincol2}{$\rightarrow$ attribuut}}\\
\end{frame}


\begin{frame}
\frametitle{Als we data samenvoegen of opsplitsen\\worden dimensies attributen en vice versa.}
\begin{center}
\tiny
\begin{tblr}{
	width=.8\linewidth,
	colspec={X[l]X[c]},
	column{1} = {fg=textcol,bg=maincol3},
	row{1} = {fg=white,bg=maincol2},
	hline{1,Z} = {maincol1},
	vline{1,Z} = {maincol1},
	hspan=minimal
	}
Conceptuele definitie & Cijfer \\
Aantal inwoners in Aalst in 2019 & 86\,445 \\
Aantal inwoners in Aalst in 2020 & 87\,382 \\
Aantal inwoners in Aalter in 2019 & 28\,906 \\
Aantal inwoners in Aalter in 2020 & 27\,394 \\
\ldots & \ldots \\
Aantal inwoners in Zwijndrecht in 2019 & 19\,056 \\
Aantal inwoners in Zwijndrecht in 2020 & 21\,343 \\
\end{tblr}
\end{center}
\vfill\mbox{}\\
Concept 1: Parameter \textcolor{maincol2}{$\rightarrow$ attribuut}\\
Concept 2: Gemeente \textcolor{maincol2}{$\rightarrow$ dimensie}\\
\textbf{Concept 3: Jaartal \textcolor{maincol2}{$\rightarrow$ dimensie}}\\
\end{frame}




\begin{frame}
\begin{small}\color{maincol2} 
\begin{tblr}{width=\linewidth,colspec={lX[l]}}
\textbf{Oefening:} & Wat zijn de attribuutconcepten en dimensies in deze cijfertabel? \\
\end{tblr}
\end{small}
\par
\medskip
\par 
\begin{tiny}
\begin{tblr}{
	width=\linewidth,
	colspec={X[c]X[c]X[c]X[c]},
	column{1-2} = {fg=textcol,bg=maincol3},
	row{1} = {fg=white,bg=maincol1},
	row{2} = {fg=white,bg=maincol2},
	hline{1,Z} = {maincol1},
	vline{1,Z} = {maincol1},
	hspan=minimal
	}
\SetCell[c=4]{l} Aantal en percentage inwoners per geslacht in Vlaamse gemeenten in 2019.\\
Gemeente & Geslacht & Aantal & Percentage \\
Aartselaar & man & 7089 & 49.6 \\
Aartselaar & vrouw & 7204 & 50.4 \\
Antwerpen & man & 262921 & 50.0 \\
Antwerpen & vrouw & 263014 & 50.0 \\
Boechout & man & 6506 & 49.0 \\
\ldots & \ldots & \ldots & \ldots \\
\end{tblr}
\par
\vfill
\par
\onslide<2->{
	\begin{small}\color{maincol2}
	\textbf{Antwoord:}
	\end{small}
	\par
	\smallskip
	\par
	\begin{itemize}\scriptsize
	\item attribuutconcepten: jaartal (2019), eenheid (inwoners)
	\item dimensies: parameter (aantal/percentage), gemeente, geslacht
	\end{itemize}
	}
\end{tiny}
\end{frame}




%\begin{frame}
%\frametitle{\onslide<2->{De afbakening van concepten is arbitrair.}}
%\vfill
%\begin{small}\color{maincol2} 
%\textbf{Oefening:} Wat zijn de concepten in volgende conceptuele definitie?
%\end{small}
%\par
%\smallskip
%\par 
%\begin{scriptsize} 
%\makebox[0pt][r]{``}het aantal inwoners in Antwerpen per bevolkingsgroep (mannen, vrouwen, 0- tot 18-jarigen, 19- tot 64- jarigen, 65+'ers, mensen met Belgische nationaliteit, mensen met buitenlandse nationaliteit,\ldots'')
%\end{scriptsize}
%\par
%\vfill
%\par
%\onslide<2->{
%	\begin{small}\color{maincol2}
%	\textbf{Antwoord:}
%	\end{small}
%	\par
%	\smallskip
%	\par
%	\begin{itemize}\scriptsize
%	\item parameter, gemeente, geslacht, leeftijd, nationaliteit,\ldots\ of
%	\item parameter, gemeente, bevolkingsgroep
%	\end{itemize}
%	}
%\end{frame}





\begin{frame}
\frametitle{Een cijfertabel heeft ook\\een conceptuele en operationele definitie.\strut}
\begin{center}
\tiny
\begin{tblr}{
	width=.8\linewidth,
	colspec={X[l]X[c]},
	column{1} = {fg=textcol,bg=maincol3},
	row{1} = {fg=white,bg=maincol2},
	hline{1,Z} = {maincol1},
	vline{1,Z} = {maincol1},
	hspan=minimal
	}
Conceptuele definitie & Cijfer \\
Aantal inwoners in Aalst in 2019 & 86\,445 \\
Aantal inwoners in Aalter in 2019 & 28\,906 \\
Aantal inwoners in Aarschot in 2019 & 30\,115 \\
\ldots & \ldots \\
Aantal inwoners in Zwijndrecht in 2019 & 19\,056 \\
\end{tblr}
\end{center}
\vfill\mbox{}\\
\begin{scriptsize}
\begin{tabularx}{\linewidth}{lL}
Conceptuele definitie: &\makebox[0pt][r]{``}Aantal inwoners per gemeente in 2019'' \\[3mm]
Operationele definitie: & \makebox[0pt][r]{``}De grootte van de wettelijke bevolking op 1 januari 2019 per gemeente volgens de NIS‐code‐indeling in 2019. De wettelijke bevolking omvat\ldots''
\end{tabularx}
\end{scriptsize}
\end{frame}


\begin{frame}
\frametitle{De operationele definitie bepaalt duidelijk\\hoeveel cijfers de tabel bevat.\strut}
\begin{center}
\tiny
\begin{tblr}{
	width=.8\linewidth,
	colspec={X[l]X[c]},
	column{1} = {fg=textcol,bg=maincol3},
	row{1} = {fg=white,bg=maincol2},
	hline{1,Z} = {maincol1},
	vline{1,Z} = {maincol1},
	hspan=minimal
	}
Conceptuele definitie & Cijfer \\
Aantal inwoners in Aalst in 2019 & 86\,445 \\
Aantal inwoners in Aalter in 2019 & 28\,906 \\
Aantal inwoners in Aarschot in 2019 & 30\,115 \\
\ldots & \ldots \\
Aantal inwoners in Zwijndrecht in 2019 & 19\,056 \\
\end{tblr}
\end{center}
\vfill\mbox{}\\
\begin{scriptsize}
\begin{tabularx}{\linewidth}{lL}
Conceptuele definitie: &\makebox[0pt][r]{``}Aantal inwoners per gemeente'' \\[3mm]
Operationele definitie: & \makebox[0pt][r]{``}De grootte van de wettelijke bevolking op 1 januari 2019 per \alert{gemeente volgens de NIS‐code‐indeling in 2019}. De wettelijke bevolking \ldots'' \\
\end{tabularx}
\end{scriptsize}
\begin{flushright}
$\Rightarrow$ 300 cijfers
\end{flushright}
\end{frame}































\section{Statistiek als een tabelreeks}


\begin{frame}
\frametitle{Een tabelreeks combineert cijfertabellen\\met overlappende definities.}
\includegraphics[width=\linewidth]{figures/tabelreeks/tabelreeks}
\end{frame}


\begin{frame}
\frametitle{Een tabelreeks heeft ook\\een conceptuele en operationele definitie.}
\begin{center}
\includegraphics[width=.5\linewidth]{figures/tabelreeks/tabelreeks}
\end{center}
\vfill\mbox{}\\
\begin{scriptsize}
\begin{tabularx}{\linewidth}{lL}
Conceptuele definitie: &\makebox[0pt][r]{``}Aantal inwoners naar geslacht en leeftijd in 2019'' \\[5mm]
Operationele definitie: & 
\begin{minipage}[t]{\linewidth}
\makebox[0pt][r]{``}De grootte van de wettelijke bevolking op 1 januari 2019 per gemeente volgens de NIS‐code‐indeling in 2019 volgens 
\begin{itemize}
\item Geslacht zoals geregistreerd in het Rijksregister (man, vrouw)
\item Leeftijd zoals geregistreerd in het Rijksregister, in vier groepen (0-18 jaar, 19-35 jaar, 36-64 jaar, 65+ jaar)
\end{itemize}
De wettelijke bevolking omvat\ldots''
\end{minipage}
\end{tabularx}
\end{scriptsize}
\end{frame}


\begin{frame}
\frametitle{De operationele definitie bepaalt duidelijk\\welke tabellen de reeks bevat}
\begin{center}
\includegraphics[width=.5\linewidth]{figures/tabelreeks/tabelreeks}
\end{center}
\vfill\mbox{}\\
\begin{scriptsize}
\begin{tabularx}{\linewidth}{lL}
Conceptuele definitie: &\makebox[0pt][r]{``}Aantal inwoners naar geslacht en leeftijd in 2019'' \\[3mm]
Operationele definitie: & 
\begin{minipage}[t]{\linewidth}
\makebox[0pt][r]{``}De grootte van de wettelijke bevolking op 1 januari 2019 per gemeente volgens de NIS‐code‐indeling in 2019 \textcolor{maincol2}{volgens} 
\begin{itemize}
\item \textcolor{maincol2}{Geslacht zoals geregistreerd in het Rijksregister (man, vrouw)}
\item \textcolor{maincol2}{Leeftijd zoals geregistreerd in het Rijksregister, in vier groepen (0-18 jaar, 19-35 jaar, 36-64 jaar, 65+ jaar)}
\end{itemize}
De wettelijke bevolking omvat\ldots''
\end{minipage}
\end{tabularx}
\end{scriptsize}
\end{frame}



\begin{frame}
\frametitle{Attributen kunnen op verschillende niveaus\\worden gedefinieerd.}
\vfill
\point{Attribuut van een tabelreeks}
\explanation{bv. gemeenschappelijke bron}
\vfill
\point{Attribuut van een cijfertabel}
\explanation{bv. attribuutconcepten zoals jaartal}
\vfill
\point{Attribuut van groep cijfers}
\explanation{bv. afwijkende operationalisering}
\vfill
\point{Attribuut van één enkel cijfer}
\explanation{bv. een fout}
\vfill
\end{frame}


















\end{document}