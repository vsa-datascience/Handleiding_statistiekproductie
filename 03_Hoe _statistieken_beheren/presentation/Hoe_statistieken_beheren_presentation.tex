% !TEX program = xelatex

\documentclass{beamer}
\usetheme[SV]{VSA}

\usepackage{tikz}
\usepackage{tabularray,tabularx}
\usepackage[framemethod=TikZ]{mdframed}

\newcommand\point[1]{{\bfseries\color{softsteelblue}#1\par}}
\newcommand\explanation[1]{{\footnotesize\quad#1\par}}



\begin{document}


\title{Hoe statistieken beheren?}
\author{Jorre Vannieuwenhuyze}
\date{}
%\date{13 oktober 2025}
%\institute{Stuurgroep transitietraject}

\titleframe




\begin{frame}
\includegraphics[width=\linewidth]{pictures/bevolkingsgrootte}
\end{frame}

\begin{frame}
\includegraphics[width=\linewidth]{pictures/bevolkingsgroei}
\end{frame}

\begin{frame}
\includegraphics[width=\linewidth]{pictures/afgevlaktebevolkingsgroei}
\end{frame}

\begin{frame}
\frametitle{We bewaren statistieken op operationeel niveau.}
\vfill
\scriptsize
\begin{tblr}{
	width=\linewidth,
	colspec={X[c]X[c]X[c]},
	row{1} = {fg=white,bg=deepslateblue},
	row{2} = {fg=white,bg=softsteelblue},
	rows = {rowsep=1pt},
	hline{1,Z} = {deepslateblue},
	vline{1,Z} = {deepslateblue},
	}
\SetCell[c=3]{halign=l} Bevolkingsgrootte &&\\
datum & gemeente & waarde \\
01-01-2019 & Aalst & 86\,445 \\ 
01-01-2019 & Aalter & 28\,906 \\
\ldots & \ldots & \ldots \\
\end{tblr}
\vfill
\pause
\begin{tblr}{
	width=\linewidth,
	colspec={X[c]X[c]X[c]},
	row{1} = {fg=white,bg=deepslateblue},
	row{2} = {fg=white,bg=softsteelblue},
	rows = {rowsep=1pt},
	hline{1,Z} = {deepslateblue},
	vline{1,Z} = {deepslateblue},
	}
\SetCell[c=3]{halign=l} Bevolkingsgroei &&\\
jaar & gemeente & waarde \\
2019 & Aalst & 273 \\ 
2019 & Aalter & -381 \\
\ldots & \ldots & \ldots \\
\end{tblr}
\vfill
\pause
\begin{tblr}{
	width=\linewidth,
	colspec={X[c]X[c]X[c]X[c]},
	row{1} = {fg=white,bg=deepslateblue},
	row{2} = {fg=white,bg=softsteelblue},
	rows = {rowsep=1pt},
	hline{1,Z} = {deepslateblue},
	vline{1,Z} = {deepslateblue},
	}
\SetCell[c=4]{halign=l} Afgevlakte bevolkingsgroei &&&\\
startdatum & einddatum  & gemeente & waarde \\
01-01-2017 & 31-12-2022 & Aalst & 123 \\ 
01-01-2017 & 31-12-2022 & Aalter & -231 \\
\ldots & \ldots & \ldots \\
\end{tblr}
\vfill
\end{frame}














%\begin{frame}
%\frametitle{Concepten kunnen worden gebruikt\\voor aparte cijferreeksen of één grote cijferreeks.}
%\tiny
%\begin{tblr}{
%	width=0.45\linewidth,
%	colspec={X[l]},
%	row{1,4,7,10,13} = {fg=white,bg=deepslateblue},	
%	row{2,5,8,11,14} = {fg=charcoalgray,bg=white},
%	row{3,6,9,12,15} = {rowsep=-2pt},
%	vline{1,Z} = {1-2,4-5,7-8,10-11,14-15}{deepslateblue},
%	hline{3,6,9,12,15} = {deepslateblue}	
%	}
%Aantal inwoners in Vlaanderen \\
%6\,821\,770 \\\\
%Percentage Vlamingen met gevorderde digitale vaardigheden \\
%26,1 \\\\
%Aantal zeugen in Vlaanderen \\
%339\,103 \\\\
%Ondergrens 95\%-CI gemiddelde vertrouwen in provinciale overheid \\
%2,34 \\\\
%\ldots \\
%\ldots \\
%\end{tblr}
%\hfill
%\begin{tblr}{
%	width=0.45\linewidth,
%	colspec={X[l]Q[c]},
%	rows = {valign=m},
%	column{1} = {fg=white,bg=softsteelblue},
%	row{1} = {fg=white,bg=deepslateblue},
%	hlines = {deepslateblue},
%	vline{1,Z} = {deepslateblue}
%	}
%Parameter & Cijfer \\
%Aantal inwoners in Vlaanderen & 6\,821\,770 \\
%Percentage Vlamingen met gevorderde digitale vaardigheden & 26,1 \\
%Aantal zeugen in Vlaanderen & 339\,103 \\
%Ondergrens 95\%-CI gemiddelde vertrouwen in provinciale overheid & 2,34 \\
%\ldots & \ldots \\
%\end{tblr}
%\end{frame}



\begin{frame}
\frametitle{Kenmerken kan je combineren in één dimensie\\of gebruiken als aparte dimensies.}
\tiny
\begin{tblr}{
	width=.45\linewidth,
	colspec={ccc},
	column{1-2} = {fg=charcoalgray,bg=lightskyblue},
	row{1} = {fg=white,bg=deepslateblue},
	row{2} = {fg=white,bg=softsteelblue},
	hline{Z} = {deepslateblue},
	vline{1,Z} = {deepslateblue}
	}
\SetCell[c=3]{l} Bevolkingsaantal 2019 \\
Gemeente & Parameter & Observatie \\
A & totaal aantal & 300 \\
A & aantal vrouwen & 180 \\
A & percentage vrouwen & 60 \\
A & aantal mannen & 120 \\
A & percentage mannen & 40 \\
B & totaal aantal & 500 \\
B & aantal vrouwen & 100 \\
B & percentage vrouwen & 20 \\
B & aantal mannen & 400 \\
B & percentage mannen & 80 \\
\end{tblr}
\hfill
\begin{tblr}{
	width=.45\linewidth,
	colspec={cccc},	
	column{1-3} = {fg=charcoalgray,bg=lightskyblue},
	row{1} = {fg=white,bg=deepslateblue},
	row{2} = {fg=white,bg=softsteelblue},
	hline{Z} = {deepslateblue},
	vline{1,Z} = {deepslateblue}
	}
\SetCell[c=3]{l} Bevolkingsaantal 2019 \\
Gemeente & Geslacht & Parameter & Observatie \\
A & totaal  & aantal     & 300 \\
A & vrouwen & aantal     & 180 \\
A & vrouwen & percentage & 60  \\
A & mannen  & aantal     & 120 \\
A & mannen  & percentage & 40  \\
B & totaal  & aantal     & 500 \\
B & vrouwen & aantal     & 100 \\
B & vrouwen & percentage & 20  \\
B & mannen  & aantal     & 400 \\
B & mannen  & percentage & 80  \\
\end{tblr}
\end{frame}



\begin{frame}
\frametitle{Concepten in een statistiekreeks\\kunnen arbitrair worden geherdefinieerd}
\tiny
\begin{tblr}{
	width=.45\linewidth,
	colspec={cccc},
	column{1-3} = {fg=charcoalgray,bg=lightskyblue},
	row{1} = {fg=white,bg=deepslateblue},
	row{2} = {fg=white,bg=softsteelblue},
	hline{Z} = {deepslateblue},
	vline{1,Z} = {deepslateblue}
	}
\SetCell[c=3]{l} Bevolkingsaantal 2019 \\
Gemeente & Geslacht & Leeftijd & Observatie \\
A & mannen  & totaal     & 90 \\
A & vrouwen & totaal     & 130 \\
A & totaal  & 18-30 jaar & 60 \\
A & totaal  & 31-65 jaar & 120 \\
A & totaal  & 66+ jaar   & 40 \\
B & mannen  & totaal     & 350 \\
B & vrouwen & totaal     & 150 \\
B & totaal  & 18-30 jaar & 20 \\
B & totaal  & 31-65 jaar & 400 \\
B & totaal  & 66+ jaar   & 80 \\
\end{tblr}
\hfill
\begin{tblr}{
	width=.45\linewidth,
	colspec={ccc},
	column{1-2} = {fg=charcoalgray,bg=lightskyblue},
	row{1} = {fg=white,bg=deepslateblue},
	row{2} = {fg=white,bg=softsteelblue},
	hline{Z} = {deepslateblue},
	vline{1,Z} = {deepslateblue}
	}
\SetCell[c=3]{l} Bevolkingsaantal 2019 \\
Gemeente & Bevolkingsgroep & Observatie \\
A & mannen  & 90 \\
A & vrouwen & 130 \\
A & 18-30 jarigen & 60 \\
A & 31-65 jarigen & 120 \\
A & 66+ jarigen   & 40 \\
B & mannen   & 350 \\
B & vrouwen & 150 \\
B & 18-30 jarigen & 20 \\
B & 31-65 jarigen & 400 \\
B & 66+ jarigen   & 80 \\
\end{tblr}
\end{frame}




%\begin{frame}
%\frametitle{Richtlijn 1}
%Cijfers die enkel verschillen in geografische indeling worden gecombineerd in één statistiekreeks waarbij deze geografische indeling als dimensie wordt opgevat. 
%\end{frame}
%
%\begin{frame}
%\frametitle{Richtlijn 2}
%Cijfers die alleen verschillen in tijdsperiode worden samengevoegd in één statistiekreeks, met de tijdsperiode als dimensie.
%\end{frame}
%
%\begin{frame}
%\frametitle{Richtlijn 3}
%Cijfers die op dezelfde manier worden berekend en uit dezelfde bron worden afgeleid, worden zoveel mogelijk gecombineerd in één reeks.
%\end{frame}
%
%\begin{frame}
%\frametitle{Richtlijn 4}
%Statistische parameters zoals frequenties en gemiddelden worden zo veel mogelijk opgesplitst in aparte reeksen.\\
%Uitzondering: Parameters die op een natuurlijke manier bij elkaar horen.
%\end{frame}
%
%\begin{frame}
%\frametitle{Richtlijn 5}
%Binnen een statistiekreeks worden de verschillende dimensies zo goed mogelijk volledig met elkaar gekruist.
%Anders herdefiniëren.
%\end{frame}
%
%\begin{frame}
%\frametitle{Richtlijn 6}
%Cijfers gebaseerd op meerdere basisreeksen uit verschillende databronnen, worden in aparte reeksen ondergebracht.
%\end{frame}
%
%\begin{frame}
%\frametitle{Richtlijn 7}
%Alle concepten worden verzameld in een centrale conceptenlijst die wordt beheerd door een toegewezen team. 
%\end{frame}
%
%\begin{frame}
%\frametitle{Statistiekreeksen kunnen verschillende versies hebben.}
%\begin{itemize}
%\item De operationele definitie van een statistiekreeks wordt aangepast omdat dimensies veranderen.
%\item De operationele definitie wordt aangepast in lijn met nieuwe gebruikersbehoeften.
%\item Er werd een fout ontdekt in de berekening van cijfers en deze fout wordt gecorrigeerd.
%\end{itemize}
%\end{frame}






%\section{Hoe statistieken beheren?}
%
%
%\begin{frame}
%\framesubtitle{We definiëren datastructuren\\per cijfertabel}
%\begin{itemize}
%\item datastructuur = data structuur definitie + codelijsten (SDMX)
%\item datastructuur afgeleid uit één algemeen datamodel
%\item datastructuur = overzicht concepten (incl. tabelreeks en cijfertabel)
%\end{itemize}
%\end{frame}
%
%% ->  tabelreeks en cijfertabel zijn ook concepten.
%
%\begin{frame}
%\frametitle{Er zijn twee soorten concepten}
%\vfill
%\pause
%\strong{Gedeelde concepten} over statistiekreeksen heen:\\
%\begin{itemize}
%\item jaartal
%\item gemeente
%\item geslacht
%\item \ldots
%\end{itemize}
%$\rightarrow$ we proberen deze te harmoniseren
%\pause
%\vfill
%Het \strong{restconcept} beschrijft de rest van de operationele definitie:\\
%bv. ``De grootte van de wettelijke bevolking op 1 januari. De wettelijke bevolking omvat \ldots''
%\vfill
%\end{frame}











%%%% SOORTEN STATISTIEKEN

%\begin{frame}
%\frametitle{Statistieken op de VSP-lijst\\krijgen bijzondere aandacht}
%\begin{mdframed}[
%	roundcorner=5pt,
%	linecolor=deepslateblue,
%	linewidth=1pt,
%	backgroundcolor=mistgray,
%	]
%Vlaamse Openbare Statistieken\\[5pt]
%\begin{small}
%\begin{minipage}{.48\linewidth}
%\begin{mdframed}[
%	roundcorner=3pt,
%	linecolor=softsteelblue,
%	linewidth=1pt,
%	backgroundcolor=lightskyblue,
%	]
%\begin{minipage}[t][45mm]{\linewidth}
%Statistieken op VSP-lijst%
%\onslide<2->{%
%	\begin{itemize}
%	\footnotesize
%	\setlength{\itemsep}{0pt}       
%	\setlength{\leftskip}{-10pt} 
%	\item metadatafiche in orde
%	\item datastructuren in orde
%	\item punctuele officiële ontsluiting
%	\item recurrent geplande kwaliteitsevaluatie 
%	\end{itemize}
%	}
%\end{minipage}
%\end{mdframed}
%\end{minipage}
%\hfill
%\begin{minipage}{.48\linewidth}
%\begin{mdframed}[
%	roundcorner=3pt,
%	linecolor=softsteelblue,
%	linewidth=1pt,
%	backgroundcolor=lightskyblue,
%	]
%\begin{minipage}[t][45mm]{\linewidth}
%Andere statistieken%
%\onslide<3->{%
%	\begin{itemize}
%	\footnotesize
%	\setlength{\itemsep}{-2pt}       
%	\setlength{\leftskip}{-10pt} 
%	\item metadatafiche in opbouw
%	\item experimentele statistieken
%	\item statistieken voor herziening
%	\item statistiek voor publicatie
%	\end{itemize}
%	}
%\end{minipage}
%\end{mdframed}
%\end{minipage}
%\end{small}
%\end{mdframed}
%\end{frame}
%
%
%
%
%
%\begin{frame}
%\frametitle{\strut Een experimentele statistiek\\heeft geen duidelijke gebruikersbehoefte.\strut}
%\begin{center}
%\Large\color{softsteelblue}\bfseries 525\,935 
%\end{center}
%\begin{tabularx}{\linewidth}{@{}>{\scriptsize}l@{ }>{\scriptsize}L@{}}
%Gebruikersbehoeften: & \makebox[0pt][l]{???}\phantom{\begin{minipage}[t]{\linewidth}
%	Decreet XX.XX verwijst naar het aantal inwoners in Vlaamse gemeenten in de context van \ldots\\
%	Agentschap Binnenlands Bestuur publiceert cijfers over het aantal inwoners in gemeenten als \ldots\\
%	Academische onderzoekers vragen cijfers over het aantal inwoners in gemeenten om onderzoek te voeren over \ldots\\
%	\end{minipage}} \\
%Conceptuele definitie: &  Aantal inwoners van de gemeente Antwerpen in 2019 \\ 
%Operationele definitie: & De grootte van de wettelijke bevolking op 1 januari 2019 in de gemeente Antwerpen (Statbel NIS-code 11002 in 2019). De data worden aangeleverd door Statbel op basis van het Rijkregister van de natuurlijke personen. De wettelijke bevolking verwijst naar \ldots \\
%Kwaliteitsevaluatie: & \makebox[0pt][l]{???}\phantom{De gebruikers hebben eerder nood aan de grootte van de verblijvende bevolking in plaats van de wettelijke bevolking.} \\
%\end{tabularx}
%\end{frame}
%
%\begin{frame}
%\frametitle{Een experimentele statistiek\\heeft geen duidelijke gebruikersbehoefte.}
%\begin{center}
%\Large\color{softsteelblue}\bfseries 525\,935 
%\end{center}
%\begin{tabularx}{\linewidth}{@{}>{\scriptsize}l@{ }>{\scriptsize}L@{}}
%\onslide<1->{ Gebruikersbehoeften: & ? \\ }
%\onslide<1->{Conceptuele definitie: &  Aantal inwoners van de gemeente Antwerpen in 2019 \\ }
%\onslide<1->{Operationele definitie: & De grootte van de wettelijke bevolking op 1 januari 2019 in de gemeente Antwerpen (Statbel NIS-code 11002 in 2019). De data worden aangeleverd door Statbel op basis van het Rijkregister van de natuurlijke personen. De wettelijke bevolking verwijst naar \ldots} \\
%\onslide<1->{Kwaliteitsevaluatie: &  ? } \\
%\end{tabularx}
%\end{frame}








\end{document}






