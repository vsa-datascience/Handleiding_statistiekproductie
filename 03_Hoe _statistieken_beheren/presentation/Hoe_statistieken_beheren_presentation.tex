
\section{Hoe statistieken beheren?}


\begin{frame}
\framesubtitle{We definiëren datastructuren\\per cijfertabel}
\begin{itemize}
\item datastructuur = data structuur definitie + codelijsten (SDMX)
\item datastructuur afgeleid uit één algemeen datamodel
\item datastructuur = overzicht concepten (incl. tabelreeks en cijfertabel)
\end{itemize}
\end{frame}




% ->  tabelreeks en cijfertabel zijn ook concepten.

%\begin{frame}
%\frametitle{Er zijn twee soorten concepten}
%\vfill
%\pause
%\strong{Gedeelde concepten} over statistiekreeksen heen:\\
%\begin{itemize}
%\item jaartal
%\item gemeente
%\item geslacht
%\item \ldots
%\end{itemize}
%$\rightarrow$ we proberen deze te harmoniseren
%\pause
%\vfill
%Het \strong{restconcept} beschrijft de rest van de operationele definitie:\\
%bv. ``De grootte van de wettelijke bevolking op 1 januari. De wettelijke bevolking omvat \ldots''
%\vfill
%\end{frame}














