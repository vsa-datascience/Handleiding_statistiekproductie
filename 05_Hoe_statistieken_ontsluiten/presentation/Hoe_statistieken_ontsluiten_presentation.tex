% !TEX program = xelatex

\documentclass{beamer}
\usetheme[SV]{VSA}

\usepackage{tikz}
\usepackage{fontawesome5}
\usepackage{anyfontsize}
\usepackage{tabularray}

\begin{document}


\title{Basisconcepten\\openbare statistiekproductie}
\author{Jorre Vannieuwenhuyze}
\date{}
%\date{13 oktober 2025}
%\institute{Stuurgroep transitietraject}

\titleframe




\begin{frame}
\frametitle{We ontsluiten Statistieken\\op verschillende manieren.}
\begin{tikzpicture}[
	x=\linewidth,
	y=12mm,
	trim left=0,trim right=\linewidth,
	icon/.style={font=\fontsize{20}{20}\selectfont,deepslateblue,inner sep=0pt},
	label/.style={anchor=south,text=softsteelblue,font=\footnotesize,inner sep=1pt,align=center,inner sep=2pt},
	arrow/.style={-latex,lightskyblue,thick,shorten >=3pt,shorten <=3pt},
	]
\node (A) at (0,0) [icon,anchor=west]{\faTable};
\node at (A.north) [label]{brondata}; 
\pause
\node (B) at (.5,0) [icon]{\faDatabase};
\node at (B.north) [label]{VOS\\databank}; 
\draw[arrow] (A) to (B);
\pause
\node (F) at (.5,-1) [label,anchor=center]{dataplatform};
\draw[arrow] (B) to (F);
\pause
\node (C) at (1,0) [icon,anchor=east]{\faGlobe};
\node at (C.north) [label]{Cijferpagina}; 
\node (D) at (1,-1.2) [icon,anchor=east]{\includegraphics[width=2ex]{icons/powerbi.png}};
\node at (D.north) [label]{PowerBI}; 
\node (E) at (1,-2.4) [icon,anchor=east]{\ldots};
\draw[arrow] (B) to (C);
\draw[arrow] (B) to (D);
\draw[arrow] (B) to (E);
\pause
\node (G) at (.25,-3.5) [icon,anchor=west,fill=white]{\faFilePdf[regular]};
\draw[arrow] (A) |- (G);
\draw[ashgray,-latex,line width=5pt,shorten <=3pt] (G) to +(10mm,20mm);
\draw[ashgray,-latex,line width=5pt,shorten <=3pt] (G) to +(20mm,20mm);
\draw[ashgray,-latex,line width=5pt,shorten <=3pt] (G) to +(20mm,10mm);
\node at (G.north) [label]{rapporten}; 
\end{tikzpicture}
\end{frame}





\section*{Methodebreuken in tijdsreeksen}


\begin{frame}
\frametitle{Om methodebreuken te tonen,\\onderscheiden we vier situaties.}
\begin{tblr}{
	width=\linewidth,
	colspec={rX[c]X[c]},
	row{1} = {font=\tiny},
	column{1} = {font=\tiny,leftsep=0pt},
	rows = {valign=m},
	hline{2-3} = {lightskyblue},
	vline{2-3} = {lightskyblue},
	}
 & zelfde operationele definitie & verschillende operationele definities\\
\begin{tabular}[c]{@{}r@{}}zelfde\\conceptuele\\definitie\end{tabular} & 
\raisebox{\dimexpr-.5\height+.5\depth}{\phantom{\includegraphics[width=\linewidth]{graphs/VTE_vlaamse_universiteiten.pdf}}} & 
\raisebox{\dimexpr-.5\height+.5\depth}{\phantom{\includegraphics[width=\linewidth]{graphs/vroegtijdige_schoolverlaters.pdf}}} \\
\begin{tabular}[t]{@{}r@{}}verschillende\\conceptuele\\definities\end{tabular} & 
& 
\raisebox{\dimexpr-.5\height+.5\depth}{\phantom{\includegraphics[width=\linewidth,page=3]{graphs/nieuwbouwvergunningen.pdf}}} \\
\end{tblr}	
\end{frame}


\begin{frame}
\includegraphics[width=\linewidth]{graphs/VTE_vlaamse_universiteiten.pdf}
\end{frame}


\begin{frame}
\frametitle{Om methodebreuken te tonen,\\onderscheiden we vier situaties.}
\begin{tblr}{
	width=\linewidth,
	colspec={rX[c]X[c]},
	row{1} = {font=\tiny},
	column{1} = {font=\tiny,leftsep=0pt},
	rows = {valign=m},
	hline{2-3} = {lightskyblue},
	vline{2-3} = {lightskyblue},
	}
 & zelfde operationele definitie & verschillende operationele definities\\
\begin{tabular}[c]{@{}r@{}}zelfde\\conceptuele\\definitie\end{tabular} & 
\raisebox{\dimexpr-.5\height+.5\depth}{{\includegraphics[width=\linewidth]{graphs/VTE_vlaamse_universiteiten.pdf}}} & 
\raisebox{\dimexpr-.5\height+.5\depth}{\phantom{\includegraphics[width=\linewidth]{graphs/vroegtijdige_schoolverlaters.pdf}}} \\
\begin{tabular}[t]{@{}r@{}}verschillende\\conceptuele\\definities\end{tabular} & 
\onslide<2->{\raisebox{\dimexpr-.5\height+.5\depth}{\color{copperclay}\huge\faTimes}} & 
\raisebox{\dimexpr-.5\height+.5\depth}{\phantom{\includegraphics[width=\linewidth,page=3]{graphs/nieuwbouwvergunningen.pdf}}} \\
\end{tblr}	
\end{frame}


\begin{frame}
\includegraphics[width=\linewidth,page=1]{graphs/nieuwbouwvergunningen.pdf}
\end{frame}

\begin{frame}
\includegraphics[width=\linewidth,page=2]{graphs/nieuwbouwvergunningen.pdf}
\end{frame}

\begin{frame}
\includegraphics[width=\linewidth]{graphs/vroegtijdige_schoolverlaters.pdf}
\end{frame}


\begin{frame}
\frametitle{Om methodebreuken te tonen,\\onderscheiden we vier situaties.}
\begin{tblr}{
	width=\linewidth,
	colspec={rX[c]X[c]},
	row{1} = {font=\tiny},
	column{1} = {font=\tiny,leftsep=0pt},
	rows = {valign=m},
	hline{2-3} = {lightskyblue},
	vline{2-3} = {lightskyblue},
	}
 & zelfde operationele definitie & verschillende operationele definities\\
\begin{tabular}[c]{@{}r@{}}zelfde\\conceptuele\\definitie\end{tabular} & 
\raisebox{\dimexpr-.5\height+.5\depth}{{\includegraphics[width=\linewidth]{graphs/VTE_vlaamse_universiteiten.pdf}}} & 
\raisebox{\dimexpr-.5\height+.5\depth}{{\includegraphics[width=\linewidth]{graphs/vroegtijdige_schoolverlaters.pdf}}} \\
\begin{tabular}[t]{@{}r@{}}verschillende\\conceptuele\\definities\end{tabular} & 
\raisebox{\dimexpr-.5\height+.5\depth}{\color{copperclay}\huge\faTimes} & 
\raisebox{\dimexpr-.5\height+.5\depth}{\phantom{\includegraphics[width=\linewidth,page=3]{graphs/nieuwbouwvergunningen.pdf}}} \\
\end{tblr}	
\end{frame}

\begin{frame}
\includegraphics[width=\linewidth,page=3]{graphs/nieuwbouwvergunningen.pdf}
\end{frame}

\begin{frame}
\frametitle{Om methodebreuken te tonen,\\onderscheiden we vier situaties.}
\begin{tblr}{
	width=\linewidth,
	colspec={rX[c]X[c]},
	row{1} = {font=\tiny},
	column{1} = {font=\tiny,leftsep=0pt},
	rows = {valign=m},
	hline{2-3} = {lightskyblue},
	vline{2-3} = {lightskyblue},
	}
 & zelfde operationele definitie & verschillende operationele definities\\
\begin{tabular}[c]{@{}r@{}}zelfde\\conceptuele\\definitie\end{tabular} & 
\raisebox{\dimexpr-.5\height+.5\depth}{{\includegraphics[width=\linewidth]{graphs/VTE_vlaamse_universiteiten.pdf}}} & 
\raisebox{\dimexpr-.5\height+.5\depth}{{\includegraphics[width=\linewidth]{graphs/vroegtijdige_schoolverlaters.pdf}}} \\
\begin{tabular}[t]{@{}r@{}}verschillende\\conceptuele\\definities\end{tabular} & 
\raisebox{\dimexpr-.5\height+.5\depth}{\color{copperclay}\huge\faTimes} & 
\raisebox{\dimexpr-.5\height+.5\depth}{{\includegraphics[width=\linewidth,page=3]{graphs/nieuwbouwvergunningen.pdf}}} \\
\end{tblr}	
\end{frame}





\end{document}


